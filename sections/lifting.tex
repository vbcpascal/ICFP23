\section{Translation Rules and Semantics Lifting}\label{sec:tr}

In this section, we describe the most important part of our method.
We assume that there exists a sufficiently powerful host language (\STLC{} with extensions).
And a DSL will be defined by translation rules based on this host language.
We first give some concepts of translation rules and impose some requirements for it.
Then, we present the semantics lifting algorithm formally with examples.
We will illustrate two basic properties of the algorithm: correctness and abstraction.
In the proofs, why structural rules are important will be explained.
Finally, we will introduce lambda abstraction and application to refine the algorithm and proofs.

\subsection{Translation Rules}

% After defining a fully featured host language,
%  translation rules are designed to specialize the language constructs of the DSL.
% The difference between a translation rule and a macro is that it is not provided by the host language, 
%  but is an abstraction outside the language.
% The distinction between translation rules and syntactic sugar is that it not only simplifies the syntax, 
%  but also can be used to describe more operations.

A translation rule $tr$ defines a new constructor $c_n$ in meta-language, with shape 
\[ c_n~\mexp_{x_1}\cdots \mexp_{x_n} => \mexp \]
 where free expression variables in $\mexp$ must appear in LHS.
We use $LHS(tr)$ and $RHS(tr)$ to denote the expressions on either side of $=>$.
We call constructors defined by translation rules \textit{surface constructors}.
Each constructor has only one signature,
 which means parameter list and pattern-based translation are not allowed.

\begin{requirement}\label{req:tr-unique}
  Each surface constructor must be defined by a unique translation rule.
\end{requirement}

If a closed expression $\cexp$ is constructed by $c_n$, define by $tr$,
 then there exists an environment $Σ$ satisfying $Σ(LHS(tr))=\cexp$,
 and $Σ(RHS(tr))$ is called \textit{one-step translation} (or desugaring), 
 written as $\ds{\cexp}$.
\textit{Total translation} is also pretty useful,
 which translates expressions of surface constructors recursively.
But if a rule translates surface constructor $c_n$ to itself, 
 i.e., a recursive translation rule,
 the full translation is not terminated.
Therefore, we propose the following requirements.

\begin{requirement}\label{req:no-recursion}
For a translation rule $tr$, constructors in $RHS(tr)$ must be those of host language, or surface constructors defined earlier.
\end{requirement}

Formally, full translation of a closed expression $\DS{\cexp}$ is defined as:
\begin{align*}
  \DS{\mexp_b} & = \mexp_b \\
  \DS{c_n~\cexp_1\cdots \cexp_n} & = \DS{\ds{c_n~\cexp_1\cdots \cexp_n}} & \text{if $c_n$ is a surface constructor} \\
  \DS{c_h~\cexp_1\cdots \cexp_n} & = c_h~\DS{\cexp_1}\cdots\DS{\cexp_n} & \text{if $c_h$ is a host constructor} 
\end{align*}

For example, given an expression $\<true>~\<and>~(\<false>~\<or>~\<true>)$ of \textsc{Bool}, then:
\begin{align*}
  \ds{\<true>~\<and>~(\<false>~\<or>~\<true>)} & = 
    \<if>~\<true>~(\<false>~\<or>~\<true>)~\<false> \\
  \DS{\<true>~\<and>~(\<false>~\<or>~\<true>)} & = 
    \<if>~\<true>~(\<if>~\<false>~\<true>~\<true>)~\<false>
\end{align*}

The full translation can be naturally extended to bones by fully translating all the expressions recursively.
We use $\DB{b}$ to express the full translation on a bone and omit its formal definition.

In addition, we assume that all the meta-functions and full translations are commutable.
The meta-functions we have used all satisfy this assumption.

\begin{assumption}\label{asm:fun-ds}
  \begin{gather*}
    f_p(\mexp_1\cdots \mexp_n)=v \quad\miff\quad f_p(\DS{\mexp_1}\cdots\DS{\mexp_n})=\DS{v} \\
    f_m(\mexp_1\cdots \mexp_n)\rr{X} v \quad\miff\quad f_m(\DS{\mexp_1}\cdots\DS{\mexp_n}) \rr{X} \DS{v} \\
  \end{gather*}
\end{assumption}

\subsubsection{Variable Scope in Translation Rules}

Pombrio et al. \cite{infer-scope} provide a method to infer scope though syntactic sugar.
They add restrictions to the syntactic sugar to keep the scope safe.
We follow the conclusion:

\begin{requirement}\label{req:close}
  A translation rule must be \textit{closed}: 
    any variable used in the RHS must be bound locally (by lambda abstraction or let binding).
\end{requirement}

The requirement prohibits external variables capture.
Also, locally bound variables must be explicitly parameterized for use in other parameters.

\begin{example}
The definition of two translation rules and their sample programs are shown below:
\[
  \begin{array}{cl|cl}
    % \text{Translation Rules} & & \text{Sample Programs} \\
    \mathit{leaked}~e => \<let>~x:\<int> =1~\<in>~e
    & \text{(\ding{51})}
    & \mathit{leaked}~(x+1)
    & \text{(\ding{55})} \\
    \mathit{captured}~e => \<if>~x~\<true>~e
    & \text{(\ding{55})}
    & \<let>~x:\<bool> =\<true>~\<in>~\mathit{captured}~\<false>
    & \text{(\ding{55})}
  \end{array}
\]

The $\mathit{leaked}$ tries to bind a constant in $x$ for use in $e$.
But $x$ referenced in $e$ is not allowed.
Thus the translation rule is actually premitted,
 but $x$ is not in scope of $e$.
As a comparison, the following definition is compliant:
\[ \mathit{leaked}'~x~e => \<let>~x:\<int> =1~\<in>~e \]
The $\mathit{captured}$ attempts to get the value of $x$ in the current environment,
 which may produce unbound identifiers after translation.
The translation rule itself is not allowed.
\end{example}

% Taking all these considerations into account, we give the following requirements.

% \subsubsection{Hygiene}

% \todo{{the title}}
% Programming languages with non-hygienic macro systems may cause the hygiene problem\cite{hygine}:
%   variable bindings are possible to be hidden by macros, and vice versa.
% For example, we define a translation rule $or'$ via $\<let>$:
% \[ e_1~or'~e_2 => \<let>~x:\<bool> =e_1~\<in>~\<if>~x~x~e_2 \]
%  where $``x"$ is a literal identifier. 
% Then the expression $\<let>~x:\<bool>=\<false>~\<in>~(\<true>~or'~x)$ will be fully translated into $\<let>~x=\<false>~\<in>~\<let>~x=\<true>~\<in>~\<if>~x~x~x$, causing an error.
% Fortunately, thanks to the requirement \ref{req:close}, the variables in the RHS must be locally bound. 
% Therefore, we can modify the names of variables safely.
% \textit{We treat literals variables bound in the RHS as mutable and always fresh.}
% We use $\texttt{@}$ to denote fresh variables, and $or'$ will be written as:
% \[ e_1~or'~e_2 => \<let>~\texttt{@x}=e_1~\<in>~\<if>~\texttt{@x}~\texttt{@x}~e_2. \]

\subsubsection{Substitution Rules}

For each language construct defined by a translation rule,
 the developer needs to define substitution rules for it.
The set of substitution rules in DSL $R'_{subst}$ is a union of the substitution rules of the host language $R_{subst}$ and these newly defined substitution rules.
Since substitution is a pure meta-function, the assumption \ref{asm:fun-ds} should be satisfied in DSL.
Intuitively, substitution and full translation are commutable.
Deriving substitution rules automatically is our future work.

\subsection{Semantics Lifting}

% In order for the DSL to support new language constructs defined by translation rules, 
%  it is necessary to provide their evaluation rules and type rules.
% In this section, a generic algorithm will be presented, 
%  to illustrate how to derive the rules for translation rules.
% For a DSL defined by multiple translation rules, 
%  we derive the evaluation and typing rules and add them to the language individually.

% And now we will give the algorithm of semantic derivation. 
Suppose a host language based on \STLC{} is defined as $L=\langle S_p, C, R_{\mE}, R_{\mT}, R_{subst}, F\rangle$,
 where $S_p=\{\Exp,\Type\}$.
We now have a set of translation rules to define DSL.
Due to requirement \ref{req:no-recursion},
 we can simplify the problem to process one translation rule one by one.
% And the language which supports the new language constructs is used as a new host language,
%  to derive evaluation rules for other translation rules.
Suppose the translation rule $tr$ has shape $c_n~exp_1\cdots exp_n => e$,
 and $sort(c_n)=\Exp$.
Our goal is to get the definition of new language $L'=\langle S_p, C', R'_{\mE}, R'_{\mT}, R'_{subst}, F\rangle$,
 which supports the new language construct $c_n$.
The definitions of $C'$ and $R_{subst}$ are obvious.
We need to define the evaluation and typing rule for $c_n$.
By the method proposed by \cite{infer-types}, the typing rule is obtained.
In the following, we will talk about how to derive the evaluation rule.

The core idea of the semantic derivation has been shown in overview:
 expand evaluation of compound expressions recursively according to the rules in $R_{\mE}$
 until all the evaluation are unexpandable.
Formally, $\EE{LHS(tr)}=\dd(\EE{RSH(tr)})$, the definition of $\dd$ is shown in Fig. \ref{fig:sd}.

\begin{figure}
  $\EE{c_n~exp_1\cdots exp_n}=\dd(\EE{e})$
  \begin{align}
    \noalign{\raggedright \hspace{2em} \bf Bone:}
    \dd(b:(br_1\cdots br_n)) & = \dd(b):(\dd(br_1)\cdots \dd(br_n)) \\
    \dd(\Let{pat}{b_1} b_2) & = \Let{pat}{\dd(b_1)} \dd(b_2) \\
    \dd(\EE{\mexp_x}) & = \EE{\mexp_x} \\  
    \dd(\EE{\mexp_b}) & = \EE{\mexp_b} \\
    \hspace{8em} \dd(\EE{c~\mexp_1\cdots \mexp_m}) 
      & = \dd(Σ(b)) \hspace{20pt} \text{if there exists $Σ$ and $\EE{\mexp}\cqq b$ in $L$,} \\
      & \hspace{61pt} \text{such that $Σ(\mexp)=c~\mexp_1\cdots \mexp_m$} \nonumber \\
    \dd(\EE{f_p(\mexp_1'\cdots \mexp_n')}) & = \EE{f_p(\mexp_1'\cdots \mexp_n')} \\
    \dd(f_m(\mexp_1'\cdots \mexp_n')) & = f_m(\mexp_1'\cdots \mexp_n') \\
    \noalign{\raggedright \hspace{2em} \bf Pure Expression:}
    \dd(\mexp') & = \mexp' \\
    \noalign{\raggedright \hspace{2em} \bf Branch:}
    \dd(pat |> b) & = pat |> \dd(b) \\ 
    \dd(pat \mid \mexp' |> b) & = pat \mid \mexp' |> \dd(b)
  \end{align}
  \caption{Semantic Derivation}
  \label{fig:sd}
\end{figure}

To illustrate the algorithm, some examples will be presented.
Example \ref{exm:nand} defines $\<nand>$ by $\<not>$ and $\<and>$.
Assume that the evaluation rules of $\<not>$ and $\<and>$ have been derived earlier.
We will observe that how $\dd$ works recursively.
For clarity, the parts applied by $\dd$ are underlined.

\begin{example}\label{exm:nand}
  $e_1~\<nand>~e_2 => \<not>~(e_1~\<and>~e_2)$:
  \begin{align*}
       & \EE{e_1~\<nand>~e_2} \\
    =~ & \dhl{\EE{\<not>~(e_1~\<and>~e_2)}} \\
    =~ & \dhl{\EE{e_1~\<and>~e_2}
            :\branch{\<true>|>\<false> \\& \<false>|>\<true>}} \\
    =~ & \dhl{\EE{e_1~\<and>~e_2}}
            :\branch{\<true>|>\dhl{\<false>} \\& \<false>|>\dhl{\<true>}} \\
    =~ & \dhl{\EE{e_1}
            :\branch{\<true>|>\EE{e_2} \\& \<false>|>\<false>}
           }:\branch{\<true>|>\<false> \\& \<false>|>\<true>} \\
    =~ & \EE{e_1}
          :\branch{\<true>|>\EE{e_2} \\& \<false>|>\<false>}
          :\branch{\<true>|>\<false> \\& \<false>|>\<true>} 
  \end{align*}  
\end{example}

We pick up the $\<let>$ defined in introduction as the second example.
% The language construct $\<let>$ is a common syntactic sugar, defined by lambda abstraction and application.
% Considering $\<let>$ as a translation rule, 
Example \ref{exm:let} presents the derivation of its evaluation rule.
% Note that the RHS of the translation rule satisfies \ref{ll:let}, and no language constructs will be generated in lambda lifting.
The last step is simplification, since that both left and right sides are constructed by $\<Abs>$ (i.e., $λ$ in language).
Simplification is sometimes useful for abstraction property, for instance in this example,
 where the evaluation rule of $\<let>$ no longer depends on lambda abstraction (even if this is allowed).
% In addition, substituion, as a meta-function, appears in the following derivation, which does not affect the expansion.
% In \todo{}, we will discuss substituion more deeply.

\begin{example}
  $\<let>~x:t=e_1~\<in>~e_2 => (λx:t.e_2)~e_1$:
  \begin{align*}
       & \EE{\<let>~x:t=e_1~\<in>~e_2} \\
    =~ & \dhl{\EE{(λx:t.e_2)~e_1}}  \\
    =~ & \dhl{\Let{λx':t'.e}{\EE{λx:t.e_2}} \Let{v_1}{\EE{e_1}} \EE{e[v_1/x']}} \\
    =~ & \Let{λx':t'.e}{\dhl{\EE{λx:t.e_2}}} \dhl{\Let{v_1}{\EE{e_1}} \EE{e[v_1/x']}} \\
    =~ & \Let{λx':t'.e}{λx:t.e_2} \Let{v_1}{\dhl{\EE{e_1}}} \dhl{\EE{e[v_1/x']}} \\
    =~ & \Let{λx':t'.e}{λx:t.e_2} \Let{v_1}{\EE{e_1}} \EE{e[v_1/x']} \\
    =~ & \Let{x'}{x} \Let{t'}{t} \Let{e}{e_2} \Let{v_1}{\EE{e_1}} \EE{e[v_1/x']}
  \end{align*}
  \label{exm:let}
\end{example}

\subsection{Properties}

With the semantic lifting algorithm, the evaluation rule of $c_n$ are generated.
We expect the rule to have the following properties:
\begin{itemize}
  \item Correctness: a closed expression $\ce$ in $L'$, should have the same value as 
         that of expression fully translated first and then evaluated in $L$.
  \item Abstraction: any constructor used in evaluation rule of $c_n$ is either a value constructor,
         or a DSL constructor.
\end{itemize}
In this section, we will presents these properties formally.

We use subscripts to distinguish in which language the computation is performed.
Then, the correctness is
\[ \EELL{\ce} \rr{X} v \quad\miff\quad \EEL{\DS{\ce}} \rr{X} \DS{v}. \]

For the proof of the correctness theorem, we will illustrate $\dd$ 
To prove the above conclusion, we first illustrate that $\dd$ does not influence the result of a closed bone $\cb$.

\begin{lemma}\label{lemma:del-sig}
  
\end{lemma}

\begin{lemma}\label{lemma:sig-bone}
  Support $\cb$ is a closed bone. Then:
  \[ \cb \rr{X} v \quad\miff\quad \dd(\cb) \rr{X} v. \]
\end{lemma}

\begin{proof}
  By induction on a derivation of $\cb \rr{X} v$. 
  The \textsc{Base}, \textsc{PureFun} and \textsc{MonadicFun} cases are immediate, since $\dd(\cb)=\cb$.
  The other cases are shown as follows:

  Case \textsc{Rec}: let $\cb=\EE{e}$, where $e$ is a closed expression.
  We consider each case separately according to $e$:
  \begin{itemize}
    \item $e$ is a base expression, or a pure meta-function invocation: $C(b)=b$;
    \item $e=c~\cexp_1\cdots \cexp_n$: we have $\EE{e} \rr{X} v$, iff $L[e] \rr{X} v$.
      Furthermore, according to the definition of $\dd$, 
      we can find an environment $Σ$ and a rule $\EE{\mexp}\cqq b_1$ in $L$, such that $Σ(\mexp) = e$, $\dd(\EE{e})=\dd(Σ(b_1))$.
      According to the definition, $Σ(b_1)=L[e]$.
      Then $\dd(\EE{e})=\dd(L[e])$.
      Since $e$ is a closed variable, $Σ(b_1)$ is a closed bone.
      By the induction hypothesis, $L[e]\rr{X} v$ iff $\dd(L[e])\rr{X} v$.
      Then, $\EE{e}\rr{X} v$, iff $\dd(\EE{e})\rr{X} v$.
  \end{itemize}

  Case \textsc{LetIn}: let $b=\Let{pat}{b_1} b_2$.
    Then $b\rr{X} v$, iff the following judgements are satisfied:  
    \[ b_1\rr{X_1}v_1 \quad Σ=match(pat,v_1) \quad Σ(b_2)\rr{X_2}v_2 \quad X=X_1;X_2, \]
     where $Σ(b_2)$ is a closed bone.
    By the induction hypothesis, we have
    \begin{align*}
      b_1\rr{X_1}v_1    & \quad\miff\quad \dd(b_1)\rr{X_1} v_1, \\
      Σ(b_2)\rr{X_2}v_2 & \quad\miff\quad \dd(Σ(b_2))\rr{X_2}v_2
    \end{align*}
    It can be proved that $\dd(Σ(b_2))=Σ(\dd(b_2))$ because $Σ$ is closed for $b_2$. 
    Therefore, we can apply rule \textsc{LetIn} to conclude
    \[ \Let{pat}{\dd(b_1)} \dd(b_2) \rr{X} v \]
    whose left-hand side equals to $\dd(b)$.

  Case \textsc{Branch}: Similar.
\end{proof}

Before the proof of correctness, we will give a stronger lemma,
 of which correctness theorem is an special case.

\begin{lemma}\label{lemma:b-db}
  Support $\cb$ is a closed bone. Then:
  \[ \cb \rr{X} v \quad\miff\quad \DB{\cb} \rr{X} \DS{v} \]
\end{lemma}

\begin{proof}
  By induction on a derivation of $\cb \rr{X} v$. 
  The \textsc{Base} case are obvious.

  Case \textsc{PureFun} and \textsc{MonadicFun}: by assumption \ref{asm:fun-ds}. 
  
  Case \textsc{Rec}: suppose that $b=\EE{e}$. There are three cases should be considered.
  \begin{itemize}    
    \item $e=c_n~\cexp_1\cdots \cexp_n$ where $c_n$ is a surface constructor defined by $c_n~\mexp_{x_1}\cdots \mexp_{x_n}=>\mexp$.
      Then $\EE{c_n~\cexp_1\cdots \cexp_n} \rr{X} v$ iff $L'[c_n~\mexp_1\cdots \mexp_n]\rr{X} v$.
      According to the definition, there exists an environment $Σ$, such that $Σ=\{ \mexp_{x_i} \mapsto \mexp_i, i\in [1..n] \}$ and $L'[c_n~\cexp_1\cdots \cexp_n]=Σ(\dd(\EE{\mexp}))$.
      Therefore:
      \begin{align*}
        & \EE{c_n~\cexp_1\cdots \cexp_n} \rr{X} v \\
        \text{iff}~~ & L'[c_n~\cexp_1\cdots \cexp_n]\rr{X} v & \text{by the rule \textsc{Rec}} \\
        \text{iff}~~ & \DB{L'[c_n~\cexp_1\cdots \cexp_n]} \rr{X} \DS{v} & \text{inductive hypothesis} \\
        \text{iff}~~ & \DB{Σ(\dd(\EE{\mexp}))} \rr{X} \DS{v}
      \end{align*}

      Because $\mexp$ is an expression defined by host language constructs,
       surface construct $c_n$ will not be used in $\dd(\EE{\mexp})$.
      Let $Σ'=\{ \mexp_{x_i} \mapsto \DS{\mexp_i}, i\in [1..n] \}$.
      Thus, $\DB{Σ(\dd(\EE{\mexp}))} = Σ'(\dd(\EE{\mexp}))$. Then:

      \begin{align*}
        & Σ'(\dd(\EE{\mexp})) \rr{X} \DS{v} \\
        \text{iff}~~ & \dd(Σ'(\EE{\mexp})) \rr{X} v & \text{lemma \ref{lemma:del-sig}} \\
        \text{iff}~~ & Σ'(\EE{\mexp})\rr{X} v & \text{lemma \ref{lemma:sig-bone}} \\
        \text{iff}~~ & \EE{Σ'(\mexp)}\rr{X} v
      \end{align*}

      Also:
      \[ Σ'(\EE{\mexp}) = \EE{Σ'(\mexp)} = \EE{\DS{Σ(\mexp)}} = \DS{\EE{Σ(\mexp)}} = \DB{\EE{c_n~\cexp_1\cdots \cexp_n}} \] 
      Thus $\EE{c_n~\cexp_1\cdots \cexp_n} \rr{X} v$ iff $\DB{\EE{c_n~\cexp_1\cdots \cexp_n}} \rr{X} \DS{v}$.
    
    \item $e=c_h~\mexp_1\cdots \mexp_n$ where $c_h$ is a host constructor. Similar.
    \item $e=f_p(\mexp_1\cdots \mexp_n)$: proved by assumption \ref{asm:fun-ds}.
  \end{itemize}

  Case \textsc{LetIn}: let $b=\Let{pat}{b_1} b_2$.
  Then $b\rr{X} v$, iff the following judgements are satisfied:  
  \[ b_1\rr{X_1}v_1 \quad Σ=match(pat,v_1) \quad Σ(b_2)\rr{X_2}v_2 \quad X=X_1;X_2, \]
  By the induction hypothesis:
  \begin{align*}
    b_1\rr{X_1}v_1    & \quad\miff\quad \DB{b_1}\rr{X_1} \DS{v_1}, \\
    Σ(b_2)\rr{X_2}v_2 & \quad\miff\quad \DB{Σ(b_2)} \rr{X_2} \DS{v_2}
  \end{align*}

  We aim to show that $Σ'=match(pat, \DS{v_1})$ and \todo{}.
  
  Case \textsc{Branch}: Similar.
\end{proof}

\begin{theorem}[Correctness]
  For any closed expression $\ce$ with sort $\Exp$ in $L'$, 
  \[ \EELL{\ce} \rr{X} v \quad\miff\quad \EEL{\DS{\ce}} \rr{X} \DS{v} \]
\end{theorem}

\begin{proof}
  % For any closed expression $\ce_h$ with sort $\Exp$ in host language $L$,
  % $\EEL{\ce_h} \rr{X} \DS{v}$ iff $\EELL{\ce_h} \rr{X} \DS{v}$. 
  Proved in lemma \ref{lemma:b-db}.
\end{proof}

\subsubsection{Abstraction}

Correctness is the foundation of the algorithm,
 while abstraction is the aim of our algorithm.
Before the proof of abstraction property,
 we first present \textit{strong abstraction theorem}.

% \begin{definition}[Strong Abstraction]
%   Consider $L'$ is defined by translation rules on the host language $L$.
%   Any constructor used in evaluation rules of $L'$ is value constructor.
% \end{definition}

\begin{theorem}[Strong Abstraction]
  If all the evaluation rules of the host language $L$ are structural,
  the derived rules are also structural.
\end{theorem}

A structural evaluation rule required that the constructors used must be value constructors.

\begin{proof}
  
\end{proof}

Strong abstraction property futher illustrates the importance of structural evaluation rules.
And abstraction allows the language constructs to be used in the rules.
Since the lambda abstraction in the translation rules has been defined as a new translation rules by lambda lifting, 
 the original translation rules are defined by this new language constructs, which is a DSL construct.
Recursively, all generated translation rules satisfy such property.
Therefore:

\begin{theorem}
  Lambda lifting maintains the abstraction.
\end{theorem}



\subsection{Preprocessing: Lambda Lifting}

We have already exemplified how lambda abstractions break the abstraction property in semantic derivation.
The solution to this problem is to expose the lambda abstractions in the translation rules 
 and define them as new language constructs of the DSL.
For the outermost lambda abstractions, the transformation is straightforward. 
The inner lambda abstractions, on the other hand, need to be extracted recursively.
The challenge here is to ensure that the translation rules are closed.
Formally, for the translation rule $tr$ which defines $c_n'$:
\newcommand{\laml}[1]{\uparrow_λ\!\!(#1)}
\begin{align}
  \laml{(λx:t.e_1)~e_2} & = (λx:t.\laml{e_1})~\laml{e_2} \label{ll:let} \\
  \laml{λx:t.e} & = λx:t.(c_n'~\exp_1..\exp_p ~ x_1..x_q) \\[-3pt]
    & \text{generating } c_n'~\exp_1..\exp_p ~ e_1..e_q => \laml{e[e_1/x_1..e_q/x_q]} \\[-3pt]
    & \text{for each expression variable } \exp_i \text{ in } e \text{ and variable } x_j \text{ not bound in } e \\
  \laml{c~exp_1\cdots exp_n} & = c~\laml{exp_1}\cdots\laml{exp_n} \\
  \laml{exp_x} & = exp_x \\
  \laml{exp_b} & = exp_b
\end{align}
where $\laml{\bullet}$ is lambda lifting transformation, with the generation of some translation rules.
The first rule states that if a lambda abstraction is applied directly in the translation rule, 
 then it does not need to be lifted because it does not break the abstraction (which can essentially be written as a $\<let>$).
The second is the main rule, which generates a new constructor $c_n'$ to describe the semantics of a lambda abstraction.
Any expression variables of $LHS(tr)$ used in $e$ (i.e., parameters of $c_n$) need to be applied by $c_n'$.
In addition, any variable $x_j$ which have been bound and captured by $e$ are also required to be included,
 by an new expression variable $e_j$, which ensures the closure of translation rules.
Note that the newly generated $c_n'$ needs to process lambda lifting recursively,
 and may generate some more translation rules.

\todo{examples}

\begin{lemma}
  Lambda lifting preserves requirements \ref{req:no-recursion} and \ref{req:close}.
\end{lemma}

