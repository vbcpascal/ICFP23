\section{Introduction}

Language-oriented programming (LOP)\cite{LOP} is solving a class of problems by designing one or more new domain-specific languages (DSLs).
To avoid building common languages constructs like loops and branches,
developers may implement DSLs on top of another general-purpose programming language (called host language).
Developers can extend the host language based on the language features, rendering programs with domain-specific forms.
Languages with features like macros and higher-order functions are frequently employed.
By specifying translation rules from the DSL to the host language, developers can obtain a DSL interpreter naturally.
This type of DSL is called embedded DSL (eDSL).

Despite eDSL can significantly reduce implementation costs, DSL users find it inconvenient due to the inevitable need to learn the host language and understand error messages.
This is called abstraction leakage, where the users writes a program in DSL and it is translated into the host language,
causing the program executed formally differently from the original program.
Much prior work has focused on how to translate information from the host language to the DSL users manually.
\todo{More related work about generate information manually}

Pombrio et al. \cite{resugar} have made a great progress in maintenance of abstraction automatically.
They proposed \textit{resugaring}, by selectively reorganizing the sequence of evaluations on the host language to the DSL according to the reverse translation rules.
But there are several practical problems in this approach.
First, programs with a lot of syntactic sugars are pretty expensive to check whether each term in the evaluation sequences can be resugared.
Second, it still contains a host language evaluator executing,
so that users cannot intuit the functionality of language constructs.
Yang et al. \cite{lazy-desg} have solved the first problem via lazy desugaring.

% In this paper, we will focus on the second problem.
% In this paper, we propose a new approach to implementing DSLs by translation rules but make it standalone.

In general, embedded DSLs treat the host language as a black box.
We take the semantics of the host language as a white box, written by the meta language,
and derive the semantics for language constructs of DSL automatically.

We follow the core ideas of inferring type rules for syntactic sugars \cite{infer-types} and extend it to the semantics.
For example, in a host language with lambda abstraction and application, we can define $\<let>$ by translation rule:
\[ e_1~\<and>~e_2 => \<if>~e_1~\<then>~e_2~\<else>~\<false>. \]
The derivation of $\<and>$ is given in Fig. \ref{fig:and}.
In DSL, an expression matching the pattern of left-hand side (LHS) evaluates to $v$, if and only if the right-hand side (RHS) evaluates to $v$ (Step 1).
Based on the rules of if-then-else, we can futher expand the evaluation of RHS.
Since if-then-else is described in two rules according to the evaluation of $e_1$,
our derivation tree have to do the same thing (Step 2).
Similarly, the rule of $\<false>$ can be applied (Step 3).
Hence, we can obtain the following conclutions:
\[
  \inference{e_1 \Da \<true> & e_2 \Da v}
  {e_1~\<and>~e_2 \Da v}
  \qquad
  \inference{e_1 \Da \<false>}
  {e_1~\<and>~e_2 \Da \<false>}
\]
It should be observed that without mentioning $\<if>$, the rules directly describe the rules of $\<and>$.
That means, the abstraction we want to maintain --- that the rules of $\<and>$ are independent of the host language --- is true.

\begin{figure}[t!]
  \[
    \inference[(Step 1) ]{%
      \inference[(Step 2) ]{%
        e_1 \Da \<true>
        & e_2 \Da v
      }
      {\<if>~e_1~\<then>~e_2~\<else>~\<false> \Da v}
    }
    {e_1~\<and>~e_2 \Da v}
    \qquad
    \inference[(Step 1) ]{%
      \inference[(Step 2) ]{%
        e_1 \Da \<false>
        & \inference[(Step 3)]{}{\<false> \Da \<false>}
      }
      {\<if>~e_1~\<then>~e_2~\<else>~\<false> \Da \<false>}
    }
    {e_1~\<and>~e_2 \Da \<false>}
  \]
  \caption{Derivation of $\mathbf{and}$}
  \label{fig:and}
\end{figure}

It is seemingly natural but we are facing three challenges:
(1) Unlike type rules, which can typically be stated by one single rule,
many evaluation rules, such as $\<if>$, do not conform this property.
This may lead to nondeterminacy when searching rules or exponential growth in the number of rules.
(2) Most type rules are defined in a modular way,
which means the type of an expression depends on the types of subexpressions,
but the semantics may be not.
In particular, when using lambda-calculus as the host language,
lambda abstraction itself is a value.
Translation rules defined by lambda abstraction, is a value itself.
The evaluation rules of them may ruin the abstraction, like
\[ \<andf> => λx:\<bool>,y:\<bool>.~\<if>~x~\<then>~y~\<else>~\<false>. \]
(3) Since the rules of application include substitution,
we must specify the behavior of the expression containing substitution in semantic derivation.

To address challenge (1),
we introduce a variant of Skeleton \cite{skeleton} as meta-language to describe semantics.
Make sure that each language structure is evaluated according to unique rules,
in order to guarantee the determinacy of semantic derivation.
To address challenge (2),
we propose lambda lifting, to reveal the semantics of lambda abstraction.
To address challenge (3),
we show that substitution can maintain the correctness and abstraction of semantic derivation,
but we shall impose greater limitations on translation rules.

In this paper, we propose a new framework for DSL design.
We use simply-typed lambda-calculus (STLC) as the host language for its elementariness and versatility.
To increase the generality of the framework,
we provide meta-extensions (to introduce new vocabularies) and monad-extensions (to introduce side-effects) on the host language.
Then, users can specify DSL constructs by translation rules on the new extended host language.
Hence, as the focus of this paper, the framework will derive the evaluation rules and type rules for these constructs automatically, to make the DSL standalone.
All the semantics are described by the meta-language, and those of DSLs are generated.
Finally and naturally, our framework will generate interpreters based on semantics.
Our main technical contributions can be summarized as follows:

\begin{itemize}
  \item \todo{host}
  \item \todo{dsl}
  \item We present an algorithm to derive semantics for DSL constructs defined by translation rules.
        For the translation rules defined with lambda abstraction,
        we will give the lambda-lifting method.
        We will prove the correctness and abstraction of the algorithm.
  \item We give an implementation of the framework.
\end{itemize}
