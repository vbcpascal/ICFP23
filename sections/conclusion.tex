\section{Conclusion}

In this paper, we propose a \diwchanged{systematic framework} to lift semantics for domain-specific languages.
We present a \diwchanged{reasonable} set of assumptions to ensure that semantics lifting maintains correctness and satisfies abstraction.
These assumptions are related to the meta-functions, host-language semantics, and translation rules.
We have proved the correctness and abstraction of the semantics-lifting algorithm under such assumptions.
Following this idea, we implement the tool Osazone, which is shown to be \diwchanged{flexible, effective, and practical} to develop DSLs.
Osazone \diwchanged{provides} a meta-language to describe the evaluation rules of \diwchanged{host languages}.
Based on a host language, users can extend the host language to support new vocabularies and language features,
 as well as specify the DSL constructs by translation rules.
As case studies, we take a functional language and a imperative language as host languages and implement DSLs based on these two languages through translation rules. We show how these languages are implemented and illustrate that meta-functions, host languages, and translation rules we used meeet our assumptions. Eventually, we obtain correct, abstract semantics for DSLs.
% \diw{Add some experimental highlights}.