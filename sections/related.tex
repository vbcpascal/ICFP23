\section{Related Work}

Our work focuses on semantics lifting for DSLs. It is related to the work on the abstraction-leakage problem, meta-language design, and DSL implementation.

\paragraph{Abstraction-Leakage Problem.}

The work on resugaring \cite{resugar} has made significant contributions to the abstraction-leakage problem of evaluation sequences, which is closely related to our work on semantics lifting for DSLs. Resugaring is a technique that lifts the core evaluation sequence into one for the surface language. A surface language is one that contains syntactic sugars, while the original language is called the core language. These languages correspond to our DSL and the host language, respectively. Our translation rules are similar to their definition of syntactic sugars. 

In the resugaring approach, an expression in the surface language is first translated into the core language and \diwchanged{then} evaluated in the core language. Then, for each expression in the host evaluation sequence, the approach checks whether it can be resugared to the surface language using reverse translation rules, dynamically. The expressions that can be resugared are shown to users as the evaluation sequence of the surface language. The resugaring algorithm is expected to hold two desired properties: emulation and coverage. Emulation requires that each expression in the resugared surface-language evaluation sequence must be desugared into an expression in the host-language evaluation sequence. Coverage requires the reconstructed evaluation sequence to be as complete as possible \diwchanged{from the user's perspective}. In contrast, our approach maintains abstraction by making the language standalone, which is done statically by lifting \diwchanged{evaluation} rules. Our approach is more efficient than resugaring when the language contains complex syntactic sugars or a program \diwchanged{is written} with a large number of syntactic sugars. Additionally, our approach can \diwchanged{pinpoint} the language construct that causes an evaluation error, which is helpful when debugging; whereas resugaring can only provide error messages for the host language.

Another highly related work is inference of typing rules \cite{infer-types}, \diwchanged{namely type lifting}, which directly inspired our semantics-lifting algorithm. In type lifting, the typing rules of the host language are given together with the translation rules, and the typing rules of the DSL are statically derived. Our approach to deriving \diwchanged{evaluation} rules can be thought of as a generalization of type lifting.

\paragraph{Meta-language Design.}
Our meta-language is designed in the style of the skeletal semantics \cite{skeleton},
which captures the behavior of a language construct in a single rule.
Other than that, by providing different interpretations,
 the skeletal semantics can prove consistency between abstract and concrete interpretations.
K-framework \cite{rosu2010overview} and Ott \cite{Ott} also allow users to define semantics formally. 
The semantics defined in K-framework is rewrite-based, using matching logic.
For ease of use, K-framework provides an executable interpreter and a program verifier.
Ott provides a meta-language to write semantics concisely and compiles \diwchanged{the semantics definition} to Coq code.
These tools focus on the formalization and verification of the language itself.
In contrast, our meta-language design concentrates on the feasibility of semantics lifting.

\paragraph{DSL Implementation.}
There is a long history of DSL implementation \cite{MartinDSL,when-how-dsl}. Our translation rules are similar to simple pattern-based macros and syntactic sugars. 
\diwchanged{Apart} from selecting the appropriate host language based on the DSL design,
 we put more emphasis on the gradual expansion of the host language \diwchanged{with} required features \cite{MoggiMeta}.
DSLs implemented using our framework are standalone from the host language and have a self-contained set of \diwchanged{evaluation} rules; \diwchanged{thus, those DSLs are} neither a deep embedding nor a shallow embedding \cite{gibbons2014folding}.