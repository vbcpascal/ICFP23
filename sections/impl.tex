\section{Implementation}\label{sec:impl}

We have implemented a prototype of our framework for language engineer in Haskell, named \textit{Osazone}.
Osazone allows users to define a host language and design a DSL by extensions and translation rules,
 corresponding to section \ref{sec:host} and \ref{sec:dsl}.
To define a host language, as a project, Osazone takes following as input:
\begin{itemize}
  \item The sorts and language constructs of the host language;
  \item The evaluation rules, typing rules and substituion rules for language constructs;
  \item Custom meta-functions;
  \item An entrance for program evaluation or typing.
\end{itemize}
Here program entrance specifies the initial state of the environment when monad is introduced in computation.
For example, the entrance of evaluation and typing in \textsc{Ref} are respectively:
\begin{flalign*}
  \hspace{2em} & \text{Evaluation: } \mathit{runState}~(\EE{\bullet})~\mathit{empty} & \\
  & \text{Typing: } \mathit{runEnv}~(\mathit{runState}~\TT{\bullet}~\mathit{empty})~\mathit{initCtx}
\end{flalign*}

Osazone then provide an interpreter for this language, by generating Haskell code.
A program can be evaluated and typed using the interpreter.
In addition, users can provide parser (with Parsec \cite{mpc, parsec}) and pretty printer of the language,
 replacing the default ones.

To define a DSL, users need to provide:
\begin{itemize}
  \item The host language.
  \item First step: a file for meta-extensions;
  \item Second step: a file for monad extensions;
  \item Third step: a file for translation rules;
  \item A configuration file to filter language constructs of the host language preserved in the DSL.
\end{itemize}
Since monad extensions are always accompanied by meta extensions,
 meta-extensions can also be performed in monad extensions.
But these three steps of extensions are processed one by one ---
 monadic meta-functions in the first step will be lifted automatically in the second step,
 while those in the monadic extensions will not.
Translation rules are finally analyzed, generating corresponding rules.
The configuration file is designed to make the DSL standalone:
 redundant language constructs can be removed safely (because of abstraction).
Then the output of Osazone is a new project for this DSL: sorts, constructors, rules, functions and entrances.
Then an interpreter of DSL is obtained for free.

We will provide some examples to demonstrate the usefulness of our framework in the next section.
