\section{Implementation}\label{sec:impl}

We have implemented a prototype of our framework for DSL engineering in Haskell, named \textit{Osazone}.
Osazone allows users to define a host language by meta-language (Section \ref{sec:host}),
 and design a DSL by extensions (Section \ref{sec:ex}) and translation rules (Section \ref{sec:tr}).

For the meta-language design, we distinguish between two types of filters in the implementation:
 pure filters and monadic filters.
The former contains pure computation, while the latter contains side effects.
For example, $(+)$ of sort $(\Int,\Int)->\Int$ is implemented as a pure filter,
 while $plus$ of sort $(\Val,\Val)->m~\Val$ is a monadic filter.
Specifically, substitution is also seen as a pure filter, defined by a set of substitution rules.

A language is called a project in Osazone.
Osazone takes the language definition and program entrances as input.
In the implementation, the language definition contains both the evaluation rules and the typing rules.
Here program entrance specifies the initial state of the environment when monad is introduced in computation.
For example, the entrance of evaluation in \textsc{Ref} without IO is 
$\mathit{runId}~(\mathit{runState}~\EE{\bullet}~\mathit{empty})$

Osazone then provides an interpreter for this language, by generating Haskell code.
A program can be evaluated and typed using the interpreter.
% In addition, users can provide parser (with Parsec \cite{mpc, parsec}) and pretty printer of the language,
%  replacing the default ones.

To define a DSL, users need to provide the host language, 
  a file for meta-extensions (first step), 
  a file for monad extensions (second step),
  a file for translation rules (third step).
% a configuration file to filter language constructs of the host language preserved in the DSL.
Since monad extensions are always accompanied by meta extensions,
 meta-extensions can also be performed in monad extensions.
But these three steps of extensions are processed one by one.
Monadic meta-functions defined in the first step will be lifted automatically in the second step,
 while those in the monadic extensions will not.
Translation rules are finally analyzed, generating corresponding rules.
We also derive the type rules in a similar way.
Therefore, the type rules of these new language constructs are also valid.
% The configuration file is designed to make the DSL standalone:
%  redundant language constructs can be removed safely (because of abstraction).
% Language constructs for values are always preserved.
The output of Osazone is a new project for this DSL definition.
Then an interpreter of DSL is obtained for free.

We will provide an example to demonstrate the usefulness of our framework in the next section.
