\section{Overview}

\begin{figure}
  \begin{tikzpicture}[x=6pt,y=6pt,yscale=1,xscale=1]

\draw[dashed] (-5,5) -- (23,5);
\node[align=left,fill=white] at (27,5) {Abstraction\\Barrier};

\node[draw] (H) at (0,0) {\STLC};
\node[draw] (D) at (0,10) {\textsc{Bool} };

\node[align=center,text=magenta] (M1) at (18,-1) {%
  $\EE{\mathit{if}~e_1~e_2~e_3} := \EE{e_1}: \left( \begin{aligned}
    & \mathit{true} \triangleright \EE{e_2} \\
    & \mathit{false} \triangleright \EE{e_3}
  \end{aligned}\right) $};

\node[align=center,text=magenta] (M2) at (18,11) {%
  $\EE{e_1~\mathit{and}~e_2} := \EE{e_1}: \left( \begin{aligned}
    & \mathit{true} \triangleright \EE{e_2} \\
    & \mathit{false} \triangleright \mathit{false}
  \end{aligned}\right) $};

\draw[->] (D) -> node[right=2pt,align=center,fill=white,pos=0.4] {$e_1~\<and>~e_2=>$\\$\<if>~e_1~e_2~\<false>$} (H);
\draw[->,thick,magenta] (18,1) -> node[left,fill=white,pos=0.6] {lifting} (18,9);
\draw[thick,magenta] (4,4) 
      to [out=-90,in=180] (6,2.5) 
      to (16,2.5) 
      to [out=0,in=-90] (18,4);

\end{tikzpicture}

  \caption{Semantics Lifting for \textsc{Bool}}
  \label{fig:bool-layer}
\end{figure}

We start with a simple example.
% to illustrate how we derive the semantics of the DSL defined by user through the translation rules.
Consider that developers would like to implement a DSL \textsc{Bool} to describe Boolean operations,
 and select \STLC{} as the host language.
We will show 
(1) how DSL developers define \textsc{Bool} by translation rules;
(2) how our algorithm lifts the semantics for \textsc{Bool}.
(3) if the developers want to implement more constructs
 but the host language is not expressive enough,
 how the developer extend the host language.

\subsection{Define \textsc{Bool} with Translation Rules}\label{sec:ov-1}

Boolean operations like $\<and>$ can be directly defined by host language constructs $\<if>$.
A translation rule, ranged over by $tr$, consists of a left-hand side (LHS) and a right-hand side (RHS) splitted by $=>$,
 where LHS is an expression containing the new language construct and pattern variables;
 and RHS is composed of host language constructs and pattern variables.
Each pattern variable must be of a specific \textit{sort}, such as $\Exp$, $\Type$.
We use $e_i$ for $\Exp$, $t_i$ for $\Type$, $x$ for identifiers.
Then, $\<and>$ can be defined as follows:
\[ e_1~\<and>~e_2 => \<if>~e_1~e_2~\<false>. \]
Each translation rule defines a new language construct in DSL.
And we require that each such language construct is determined by a unique translation rule.

\begin{figure}[t!]
  \begin{align*}
    \<not>~e        & => \<if>~e~\<false>~\<true> &
    e_1~\<or>~e_2   & => \<if>~e_1~\<true>~e_2 \\
    e_1~\<nand>~e_2 & => \<not>~(e_1~\<and>~e_2) &
    e_1~\<nor>~e_2  & => \<not>~(e_1~\<or>~e_2) \\
    \noalign{$e_1~\<xor>~e_2 => (e_1~\<and>~\<not>~e_2)~\<or>~(\<not>~e_1~\<and>~e_2)$}   
  \end{align*}
  \caption{Translation Rules for \textsc{Bool}}
  \label{fig:bool_tr1}
\end{figure}

Translation rules of the rest language constructs in \textsc{Bool} are shown in Fig. \ref{fig:bool_tr1}.
In the translation rule of $\<nand>$, DSL constructs $\<not>$ and $\<and>$ are used.
This is permitted because we can think of these translation rules as being added to the host language one by one,
 i.e. $\<nand>$ is defined based on a host language that contains $\<and>$ and $\<not>$.
But the translation rules with cyclic dependencies like recursively defined translation rules are not allowed.

\subsection{Semantics Lifting for \textsc{Bool}}

Semantics lifting, or evaluation rule derivation,
 takes the rules of host language written in meta-language and a translation rule as input,
 and outputs the evaluation rule of the new language construct written in meta-language as well.
Therefore, before discussing the semantics lifting algorithm,
 we first briefly introduce the meta-language by an example.
Consider the $\<if>$ construct with generic subexpressions denoted by $e_1$, $e_2$ and $e_3$,
 whose big-step operational semantics is defined using two separate rules.
\[
  \inference{e_1 \Da \<true> & e_2 \Da v_2}
  {\<if>~e_1~e_2~e_3 \Da v_2} \quad
  \inference{e_1 \Da \<false> & e_3 \Da v_3}
  {\<if>~e_1~e_2~e_3 \Da v_3}
\]
In our meta-language, the evaluation of $\<if>$ expression is given by a single rule:
\[
  \ctr{If}{\<if>~e_1~e_2~e_3} \cqq
  \HH{e_1}: \branch{
    \<true>  |> \HH{e_2} \\&
    \<false> |> \HH{e_3}
  }.
\]
Here, $\HH{e_1}$ identifies the evaluation of subexpression $e_1$ (hook),
 whose result is used to select a specific branch by pattern matching (after the colon).
Then, subsequent computations are processed (after the triangle).
If the value of $e_1$ is $\<true>$, then $e_2$ will be evaluated;
 and if the value is $\<false>$, then $e_3$ will be evaluated.
% All the evaluation rules can be expressed by recursive computations, meta-operations and branches.

Now it is time for our algorithm. 
For any $e_1$ and $e_2$, according to the translation, 
 the value of $e_1~\<and>~e_2$ should be the same as that of $\<if>~e_1~e_2~\<false>$. Using evaluation rules of the host language, we have
\begin{align*}
  \ctr{And}{e_1~\<and>~e_2} 
    & \cqq \HH{\<if>~e_1~e_2~\<false>} & \text{(Translation rule)} \\
    & = \HH{e_1}: \branch{
        \<true>  |> \HH{e_2} \\&
        \<false> |> \HH{\<false>}
      } & \text{(Evaluation rule of $\<if>$)} \\
    & = \HH{e_1}: \branch{
        \<true>  |> \HH{e_2} \\&
        \<false> |> \<false>
      } & \text{(Evaluation rule of $\<false>$)} 
\end{align*}
The evaluation rule of $\<and>$ has been derived, as shown in Fig. \ref{fig:bool-layer}.
The following points are worth noting:
(1) \textit{Abstraction}: the semantics of $\<and>$ is defined straightforward, and no longer needs to be expressed by $\<if>$ from the host language.
That is, even if we remove $\<if>$ from the language, the $\<and>$ works fine. 
But the use of $\<true>$ and $\<false>$ is allowed.
We assume that all the values (except lambda abstractions) in the host language can be used in the rules of the DSL.
(2) \textit{Uniqueness}: due to our requirement of uniqueness of the rules, 
 the rule applied in the derivation is always deterministic.
Also thanks to uniqueness of translation rules,
 the rules we derive for these new language constructs stay unique.

\paragraph{Lambda Lifting.}
As mentioned above, a language construct maybe be defined as a lambda abstraction in the translation rule.
For example, we can define $\<andf>$ by the following translation rule:
\[ \<andf> => λx\!:\!\<bool>.λy:\<bool>.\<if>~x~y~\<false> \]
The above method will result in a rule which breaks abstraction:
\begin{align*}
  \ctr{AndF}{\<andf>} 
    & \cqq \EE{λx\!:\!\<bool>.λy:\<bool>.\<if>~x~y~\<false>} \\
    & = λx\!:\!\<bool>.λy:\<bool>.\<if>~x~y~\<false>, 
\end{align*}
But by converting $\<andf>$ to $\<and>$, the internal computation can be exposed.
The approach of exposing parameters of the lambda abstraction in RHS,
 and transforming the rule into a set of new translation rules, is called lambda lifting.
In the above example, $\<andf>$ will be transformed into the following two rules:
\begin{align*}
  \mathit{and_f'}~e_1~e_2 & => \<if>~e_1~e_2~\<false> \\ 
  \<andf> & => λx\!:\!\<bool>.λy:\<bool>.\mathit{and_f'}~x~y.
\end{align*}
By thinking of $\mathit{andf'}$ as a DSL construct and further deriving evaluation rule for $\mathit{andf'}$,
 the abstraction property of $\<andf>$ is satisfied.
The derived evaluation rule of $\<andf>$ is defined as follows:
\[ \ctr{AndF}{\<andf>} \cqq λx\!:\!\<bool>.λy:\<bool>.\mathit{and_f'}~x~y \]
Lambda lifting plays an important role in translation rules with nested lambda abstractions on the RHS.

\subsection{Extend the host language}

Consider that developers would like to support $\<halfa>$ and $\<fulla>$ to simulate digital circuit based on the \textsc{Bool} language, named \textsc{Adder}.
For example, the expression $\<halfa>~\<true>~\<false>$ of \textsc{Adder} is expected to evaluate to $(\<false>, \<true>)$.
However, they cannot be implemented by translation rules,
 because half adder and full adder will get the pair of sum and carry as the results. 
Pairs are demanded to build compound data structures now.
Meta-extensions are used to make the host language support new types.
In this example, developers add pair and projection to the language,
% With meta-language extensions, developers can make the host language support pairs.
% Naturally, projection $\<fst>$ and $\<snd>$ are involved as $\Exp$,
%  and product $t \times t$ is involved as $\Type$.
and indicate the evaluation (and typing) rules for each newly defined language constructs.
The evaluation rules are as shown in Fig. \ref{fig:bool-meta-ex}.

\begin{figure}[t!]
  \begin{align*}
    \ctr{Pair}{(e_1,e_2)} & \cqq \lt~v_1=\EE{e_1};~\lt~v_2=\EE{e_2};~(v_1,v_2) \\
    \ctr{Fst}{\<fst>~e} & \cqq \lt~(v_1,\_)=\EE{e};~v_1 \\
    \ctr{Snd}{\<snd>~e} & \cqq \lt~(\_,v_2)=\EE{e};~v_2 
  \end{align*}
  \caption{Meta-Extension Rules for \textsc{Bool}}
  \label{fig:bool-meta-ex}
\end{figure}

Based on this host language with meta-extensions,
 $\<halfa>$ and $\<fulla>$ can be defined by translation rules.
The translation rules are shown in Fig. \ref{fig:bool_adder}.
And the evaluation rules of these language constructs can be derived using the above algorithm.

\begin{figure}[t!]
  % \begin{align*}
  %   e \in \Exp  & ::= \cdots \mid (e,e) \mid \<fst>~e \mid \<snd>~e \\
  %               & \quad \mid \<halfa>~e~e \mid \<fulla>~e~e~e \\
  %   t \in \Type & ::= \cdots \mid t\times t
  % \end{align*}
  \begin{align*}
    \<halfa>~e_1~e_2 & => (e_1~\<xor>~e_2,e_1~\<and>~e_2) \\
    \<fulla>~e_1~e_2~e_3 & => (e_1~\<xor>~e_2~\<xor>~e_3, (e_1~\<xor>~e_2)~\<or>~(e_3~\<and>~(e_1~\<xor>~e_2)))
  \end{align*}
  \caption{Translation Rules for \textsc{Adder}}
  \label{fig:bool_adder}
\end{figure}
