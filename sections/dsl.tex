\section{DSL Definition}

In this section, language extensions and language specialization will be discussed in more detail.

\subsection{Meta-Extensions}

Meta-extensions are direct extensions to the host language by
 adding new constructors $C_n$ and corresponding rules, and functions $F_n$.
This expansion has the following requirements: for each $c \in C_n$,
 (1) $sort(c) \in S_p$, i.e. $\Exp$ or $\Type$;
 (2) if $sort(c)=\Exp$, then the evaluation, typing and substitution rules should be specified.
%  (3) if there exists $e_1\cdots e_n$ such that $c~e_1\cdots c_n$,  

\paragraph{Example: integer extension.}
Consider integers are expected to be supported via literals.
New constructors, $\<Int>$ as $\Type$, $\<Lit>$, $\<Plus>$, $\<Lt>$ and $\<Eq>$ as $\Exp$, will be added.
Fig. \ref{fig:int} shows the formal definition of integer extension, 
 where $F_n=\{(+),(<)\}$, both of them are pure meta-functions.
\todo{function Substitution will be modified after adding new constructors}

\begin{figure}
  \begin{align*}
    & \<Int>:\Type \\ 
    & \<Lit>:\Int->\Exp & 
    & \EE{\<Lit>~i} \cqq \<Lit>~i \hspace{5em} \TT{\<Lit>~i} \cqq \<Int> \\
    & & & (\<Lit>~i)[e/x] = \<Lit>~i \\
    & \<Plus>:(\Exp,\Exp)->\Exp &
    & \EE{\<Plus>~e_1~e_2} \cqq \Let{\<Lit>~i_1}{\EE{e_1}} \Let{\<Lit>~i_2}{\EE{e_2}} \<Lit>~(i_1+i_2) \\
    & & & \TT{\<Plus>~e_1~e_2} \cqq \Let{\<Int>}{\TT{e_1}} \Let{\<Int>}{\TT{e_2}} \<Int> \\
    & & & (\<Plus>~e_1~e_2)[e/x]=\<Plus>~e_1[e/x]~e_2[e/x] \\
    & \<Lt>:(\Exp,\Exp)->\Exp &
    & \EE{\<Lt>~e_1~e_2} \cqq \Let{\<Lit>~i_1}{\EE{e_1}} \EE{e_2}:\branch{
        \<Lit>~i_2 \mid i_1 < i_2 |> \<True> \\&
        \<Lit>~\_ |> \<False> 
      } \\
    & & & \TT{\<Lt>~e_1~e_2} \cqq \Let{\<Int>}{\TT{e_1}} \Let{\<Int>}{\TT{e_2}} \<Bool> \\
    & & & (\<Lt>~e_1~e_2)[e/x]=\<Lt>~e_1[e/x]~e_2[e/x] \\
    & \<Eq>:(\Exp,\Exp)->\Exp &
    & \EE{\<Eq>~e_1~e_2} \cqq \Let{\<Lit>~i_1}{\EE{e_1}} \EE{e_2}:\branch{
        \<Lit>~i_2 \mid i_1 = i_2 |> \<True> \\&
        \<Lit>~\_ |> \<False> 
      } \\
    & & & \TT{\<Eq>~e_1~e_2} \cqq \Let{\<Int>}{\TT{e_1}} \Let{\<Int>}{\TT{e_2}} \<Bool> \\
    & & & (\<Eq>~e_1~e_2)[e/x]=\<Eq>~e_1[e/x]~e_2[e/x] \\
  \end{align*}
  \caption{Formal Definition of Integer Extension}
  \label{fig:int}
\end{figure}

\subsection{Monad Extensions}

As it was mentioned above, monad extension is the technique for side effects in computation without changing the original semantic definition.
By monad $m$, individual language features can be defined, such as environment, store, nondeterminism, I/O etc.
To put these together in a modular way, we use a mechanism called \textit{monad transformers}, ranged by $m_t$.
In this section, we will show the introduction of reference to STLC.
Both the change of pure functions to those including computational effects via monad (in evaluation) and the expansion of language features via monad transformer (in typing) will be explained.

\subsubsection{Evaluation with Store}
Monad extensions are often accompanied by meta-extensions. 
The basic language constructs for reference are allocation, dereferencing and assignment.
For reading convenience, we use expression with syntax like $ref~e$, $!e$ and $e_1:=e_2$.
To carry through the modification on store, 
 the signature of $\mathcal{E}$ changes from $\Exp->\Exp$,
 to $\Exp->(\mathit{State~Store})~\Exp$, abbreviated as $M_e~\Exp$,
 where $\mathit{Store}$ maps locations to values.
\[ \mathcal{E} : \Exp -> \Exp ~\wkalt{$\monad \mathit{State~Store} \as M_e$} \]
An abstract index of store is a location (used as a base sort $Loc$).
A reference will be evaluated to a location expression, constructed by $\<Loc>:Loc->\Exp$.
New (monadic) meta-functions are also necessary for evaluation rules:
\begin{itemize}
  \item $\newLoc:M_e~Loc$ returns a newly allocated location;
  \item $\store:(Loc,\Exp)->M_e~()$ updates the store to make the location contain the value;
  \item $\fetch:Loc->M_e~\Exp$ gets the value at the location from the store.
\end{itemize}
Given these meta-functions, Fig. \ref{fig:ref_eval} shows the evaluation rules of these newly defined language constructs.

\begin{figure}
  \begin{align*}
    \EE{\<Loc>~loc} & \cqq \<Loc>~loc \\
    \EE{ref~e} & \cqq \Let{v}{\EE{e}} \Let{loc}{\newLoc} \<Loc>~loc \\
    \EE{!e} & \cqq \Let{\<Loc>~loc}{\EE{e}} \fetch~loc \\
    \EE{e_1:=e_2} & \cqq \Let{\<Loc>~loc}{\EE{e}} \Let{v_2}{\EE{e_2}} \Let{()}{\store~(loc,v)} ()
  \end{align*}
  \caption{Evaluation Rules of \textsc{Ref}}
  \label{fig:ref_eval}
\end{figure}

When introducing monad, pure meta-functions keep the original definition, 
 while monadic meta-functions $f_m:(s_1\cdots s_n)->s$ need to lift to $\lifted{f_m}:(s_1\cdots s_n)->m~s$ for type correctness, defined by:
\[ \lifted{f_m}~args = \return~(f_m~args). \]

\subsubsection{Typing with Store}
If the content of a location has type $t$, then the type of the location is $Ref~t$,
 introducing new constructor $\<Ref>:\Type->\Type$.
To extend the store typing which records the association between location and their types,
 the signature of $\mathcal{T}$ changes from $\Exp->State~CtxStack~\Type$,
 to $\Exp->\mathit{ReaderT}~Store_t~(State~CtxStack)~\Type$,
 where $Store_t$ maps locations to types,
 and $ReaderT~Store_t$ is a monad transformer.
\[ \mathcal{T} : \Exp -> \Type ~\monad \mathit{State~CtxStack}, \wkalt{$\mathit{Reader~Store}_t$} \as M_t \]
Monad transformers allow us touch high-level language features, but skill keep the access to low-level details.
Now, for example, we can use both the meta-functions associated with store typing like $ask$ and the meta-functions provided by Context like $searchCtx$.
Fig. \ref{fig:ref_type} gives the typing rules of language constructs in \textsc{Ref}.

\begin{figure}
  \begin{align*}
    \TT{\<Loc>~loc} & \cqq \Let{t}{ask~loc} Ref~t \\
    \TT{ref~e} & \cqq \Let{t}{\TT{e}} Ref~t \\
    \TT{!e} & \cqq \Let{Ref~t}{\TT{e}} t \\
    \TT{e_1:=e_2} & \cqq \Let{Ref~t}{\TT{e_1}} \TT{e_2}:(t_2 \mid t = t_2 |> Unit)
  \end{align*}
  \caption{Typing Rules of \textsc{Ref}}
  \label{fig:ref_type}
\end{figure}

Likewise, pure meta-functions remains unchanged,
 and all the monadic meta-functions $f_m:(s_1\cdots s_n)->m~s$ need to lift to $\lifted{f_m}:(s_1\cdots s_n)->(m_t~m)~s$ by $\mathit{lift}$ operation\cite{todo},
 so that the type rule of lambda abstraction in \textsc{Ref} will be written as:

\begin{align*}
  \TT{λx:t.e}   & \cqq \lt~\_=\lifted{updateCtx}~x~t;~\lt~t'=\TT{e}; \\
                & \hspace{13pt} \lt~\_=\lifted{restoreCtx};~t->t'.
\end{align*}

\subsection{Translation Rules}

After defining a fully featured host language,
 translation rules are designed to specialize the language constructs of the DSL.
The difference between a translation rule and a macro is that it is not provided by the host language, 
 but is an abstraction outside the language.
The distinction between translation rules and syntactic sugar is that it not only simplifies the syntax, 
 but also can be used to describe more operations.

A translation rule $tr$ define a new constructor in meta-language,
 with shape 
\[ c_s~exp_{x_1}\cdots exp_{x_n} => exp_r, \]
 where expression variables in $exp_r$ must appear in LHS.
We use $LHS(tr)$ to express LHS, and $RHS(tr)=exp_r$ to express RHS.
We call constructors defined by translation rules \textit{surface constructor}.
If a closed expression $exp$ is constructed by $c_s$, 
 then there exists an environment $Σ$ satisfying $Σ(LHS(tr))=exp$,
 and $Σ(RHS(tr))$ is called \textit{one-step translation} (or desugaring), 
 written as $\ds(exp)$.
\textit{Total translation} is also pretty useful, 
 which expands the rules recursively. 
But if a rule translates to a constructor defined by itself, 
 i.e., a recursive translation rule, 
 then total translation is not terminated.
Therefore, we propose the following requirement.

\begin{requirement}\label{req:no-recursion}
For a translation rule $tr$, constructors in $RHS(tr)$ must be those of host language, or surface constructors defined earlier.
\end{requirement}

Formally, a total translation of expression $\DS(exp)$ is defined as:
\begin{align*}
  \DS(exp_b) & = exp_b \\
  \DS(c_s~exp_1\cdots exp_n) & = \DS(\ds(c_s~exp_1\cdots exp_n)) & \text{if $c_s$ is a surface constructor} \\
  \DS(c_h~exp_1\cdots exp_n) & = c_h~\DS(exp_1)\cdots\DS(exp_n) & \text{if $c_h$ is a host constructor} 
\end{align*}

For example, given an expression $\<true>~\<and>~(\<false>~\<or>~\<true>)$ of \textsc{Bool}, then:
\begin{align*}
  \ds(\<true>~\<and>~(\<false>~\<or>~\<true>)) & = 
    \<if>~\<true>~(\<false>~\<or>~\<true>)~\<false> \\
  \DS(\<true>~\<and>~(\<false>~\<or>~\<true>)) & = 
    \<if>~\<true>~(\<if>~\<false>~\<true>~\<true>)~\<false>
\end{align*}

\subsubsection{Variable Scope in Translation Rules}

We have very strict requirements for the scope of variables in the translation rules.
Variables defined in the rule cannot be leaked,
 and external variables may not be captured in the rule.
These translation rules are not premitted:
\begin{align*}
  \mathit{leaked}~e & => \<let>~x:\<int> =1~\<in>~e \\
  \mathit{captured}~e & => \<if>~x~\<true>~e 
\end{align*}
The former tries to bind a constant in $x$ for use in e, like $\mathit{leaked}~(x+1)$;
and the latter attempts to get the value of $x$ in the current environment, like $\<let>~x=\<true>~\<in>~\mathit{captured}~\<false>$.
From the user's point of view, it is weird to use variables that are not explicitly defined and use variables but not explicitly stated.
And in the semantic derivation, the variables that are not substituted will be stuck.
Taking all these considerations into account, we give the following requirements.

\begin{requirement}\label{req:close}
  A translation rule must be \textit{closed}: any variable used in RHS has a local binding.
\end{requirement}

\subsubsection{Hygiene Problem}

Programming languages with non-hygienic macro systems may cause the hygiene problem\cite{hygine}:
 variable bindings are possible to be hidden by macros, which we also have to face.
For example, we define a translation rule $or'$ via $\<let>$:
\[ e_1~or'~e_2 => \<let>~``x"=e_1~\<in>~\<if>~``x"~``x"~e_2 \]
where $``x"$ is a literal identifier. 
Then the expression $\<let>~x=\<false>~\<in>~(\<true>~or'~x)$ will be totally translated into $\<let>~x=\<false>~\<in>~\<let>~x=\<true>~\<in>~\<if>~x~x~x$, causing an error.
Fortunately, thanks to the requirement \ref{req:close}, the variables in the RHS must be locally bound. 
Therefore, we can modify the names of variables safely.
\textit{We treat literals variables bound in the RHS as mutable and always fresh.}
We use $@$ to denote fresh variables, and $or'$ will be written as:
\[ e_1~or'~e_2 => \<let>~@x=e_1~\<in>~\<if>~@x~@x~e_2. \]

\subsubsection{Substitution Rules}


