\section{Meta-Languages for Defining Semantics}\label{sec:host}

In this section, we define a meta language for defining language semantics, and demonstrate its use in defining the semantics of \STLC{}, a simple typed lambda calculus.
%Then we want to emphasize the importance of structural evaluation rules of host language,
 %in preparation for the semantics derivation (Section \ref{sec:m-host}).
%Note the difference between concepts of meta-language and host languages (or any language defined by meta-language) in this section.
%We use $L$ to refer to a language defined by meta-language like \STLC.
%In the remainder of this section, we will indicate such language with an $L$ to distinguish from the meta-language.
Our meta-language is based on Skeleton \cite{skeleton}, a powerful tool 
for a skeletal semantics of a language, where each skeleton describes the complete semantics behavior of a language construct. 
In Skeleton, 
a skeleton body is defined by a sequence of bones,
which are recursive computations (called hooks), filters, or branches.
Two new features of our language are
(1) We use pattern matching in place of branches and filters for branch selection;
(2) We support monads to avoid pass flow variables of states explicitly.

%Skeleton focuses on illustrating the consistency of concrete and abstract interpretations.
%The meaning of filters varies in different goals.
%But our meta-language concentrate on evaluation and typing,
% whose design is intuitively close to Haskell.
%In particular, we replace branches and merging operation with pattern matching,
% and replace filters with meta-functions.
%In addition, monad is supported to cover side effects in meta-language. % at the meta-language level.

%\subsection{Meta-Language}\label{sec:meta}

For example, the following rule defines the semantics of $\<ref>$, used to allocate a reference and initialize the cell (Types and Programming Languages \cite{tapl}, Chap. 13).
\[ \ctr{Ref}{\<ref>~e} \cqq \Let{v}{\HH{e}} \Let{l}{\mathit{newLoc}} \Let{()}{\mathit{store}(l,v)} loc~l \]
The left-hand side is specified by name of the rule, a constructor and a list of \textit{expression variables}.
Each expression variable has a \textit{sort}, which illustrates the type of the expression in the meta-language.
For example, $e$ here has sort $\Exp$.
The right-hand side can be seen as a sequence of \textit{bones}.
The result of hook $H(e)$ is bound to an expression variable $v$, whose sort is $\Val$ (a subset of $\Exp$).
$\mathit{newLoc}$ and $\<store>$ are filters, representing the primitive operations of meta-language, which are pre-defined.
We assume that any user understands these filters.
The filter $\mathit{newLoc}$ is used to allocate a new cell in storage,
 and the result is bound to $l$, whose sort is a base sort $\mathit{Location}$.
$\<store>$ puts the value $v$ into the location $l$ and always returns a unit $()$.
Instead of binding it to a variable, we check the result directly by pattern matching.
The last bone stands for the return value of this rule,
 where $loc$ is a constructor that constructs the location $l$ as an $\Val$.\footnote{Both pattern matching and results with constructors in skeleton need to be implemented via filter.
 Since we are concerned with semantics, as a concrete interpretation in Skeleton, we simplify the design here.}

Reference actually requires a global store, 
 and the filters $\mathit{newLoc}$ and $\mathit{store}$ will modify the store.
Without passing the store explicitly,
 we inscribe this side effect via monads.
For example, the output sort of this rule is $\mathit{State~Store}~\Val$,
 where $\mathit{Store}$ maps locations to values.
The output sorts of $\mathit{newLoc}$ and $\mathit{store}$ are also defined with this monad.
In Haskell, this rule can be translated as:
\[ \mathit{eval~(Ref\ e) = do~~ \{ 
    v\leftarrow eval~e;
    l\leftarrow newLoc;
    () \leftarrow store~l~v;
    return~(loc\ l) \}} \]

\subsection{Expressions}
Various \textit{expressions} in the meta-language are used to encode the language constructs in the language defined.
An expression $\mexp$ is built by \textit{base expressions}, like literals and identifiers, ranged over by $\mexp_b$;
 \textit{expression variables}, ranged over by $\mexp_x$;
 and \textit{compound expressions} by \textit{constructors}, ranged over by $c$.
Each language construct in $L$ corresponds to a constructor $c$ in meta-language.
To distinguish between various sorts of language constructs,
 we define $S$ for a finite set of \textit{sorts}, ranged over by $s$.
A sort $s$ can be either an \textit{base sort} like $\Int$, $\Id$, $\mathit{Location}$;
 or a \textit{program sort} like $\Exp$, $\Val$, and $\Type$ in \STLC.
The sort of an expression $\mexp$, written as $sort(\mexp)=s$.
A constructor $c$ has a signature $sig(c)$, which is of the form $(s_1\cdots s_n)->s$,
 where $s_1\cdots s_n$ are appropriate sorts of parameters of $c$,
 and $s$ is the sort of expressions constructed by $c$, briefly stated as sort of $c$.
A compound expression, written as $c(\mexp_1,\cdots,\mexp_n)$ or $c~\mexp_1\cdots\mexp_n$,
 is valid, if $sort(\mexp_i)=s_i$ for each $i\in [1..n]$.

We use $i$ for expressions with sort $\Int$, $x$ for $\Id$, $e$ for $\Exp$, $v$ for $\Val$, and $t$ for $\Type$.
A \textit{closed expression} $\cexp$ is an expression with no expression variables, generally refering to a concrete component in language $L$.
Let $Σ$ be an \textit{environment} mapping from expression variables to expressions. 
We write $Σ(\mexp)$ to replace all expression variables $\mexp_x\in dom(Σ)$ with $Σ(\mexp_x)$ in $\mexp$.
An environment $Σ$ is \textit{closed} for expression $\mexp$, when $Σ(\mexp)$ is a closed expression.

Furthermore, we introduce substitution $e[e'/x]$ in the meta-language.
Substitution is a special metalanguage operation.
% The substitution rules are defined according to the scope.
% Currently, we require the users to explicitly define the substitution rules.
An expression containing a substitution is also an expression and is only used in hooks.

\subsection{Rules of Meta-Language}

A rule describes the behaviour of a language construct.
A rule in meta-language has the shape $\ctr{Name}{c~\mexp_{x_1}\cdots \mexp_{x_n}} \cqq b$, 
where $\textsc{Name}$ is the name of rule,
$c$ is a constructor, 
$\mexp_{x_1}\cdots \mexp_{x_n}$ are expression variables,
and $b$ is the rule body.
Assume that a countable set of monads, ranged over by $m$.
Given a program sort $s_1$, the rules of $s_1$ have output sort $m~s_2$,
 written as $out(s_1)=m~s_2$,
 where $m$ is the monad for side effects.
When no side effects are introduced, $m$ is specified as identity monad, which can be omitted.
The rule body is composed of bones, shown in Fig. \ref{fig:body}.
We use $\mathit{fsort}(F)$ to represent the signature of filters, of the form $(s_1\cdots s_n) -> m~(s_1'\cdots s_m')$.
We require that each filter in the bone has the same monad as the output sort of the rule.
Pattern matching is used to match an expression to patterns, possibly generating new expression variable bindings.
We use $\mathit{match(pat,\mexp)}$ to indicate an attempt to match $\mexp$ with $pat$,
 getting a FAIL or an environment $Σ$ for the body of branch.

\begin{figure}
  \begin{align*}
    b \in Bone 
      & ::= \mexp & \text{(Expression)} \\
      & \quad \mid H(\mexp) & \text{(Hook)} \\
      & \quad \mid F(\mexp_1\cdots \mexp_n) & \text{(Filter)} \\
      & \quad \mid b:(br_1\cdots br_n) & \text{(Pattern Matching)} \\
      & \quad \mid \Let{pat}{b_1} b_2 & \text{(Let Binding)} \\
    br \in Branch
      & ::= pat |> b \\
    pat \in \mathit{Pattern}
      & ::= \mexp_x & \text{(New Expression Variable)} \\
      & \quad \mid c~pat_1\cdots pat_n & \text{(Compound Pattern)}
  \end{align*}
\caption{Body of Rules in Meta-Language}
\label{fig:body}
\end{figure}

A \textit{closed bone} $\cb$ is a bone without any free expression variables,
 which can be reduced to an expression.
Also, we extend the environment application to bone as $Σ(b)$, which replaces free expression variables in $dom(Σ)$.

\subsection{Language Defined by Meta-Language}
A \textit{Language} $L$ consists of a set of program sorts $S_p$;
 a set of language constructs $C$ with signatures;
 a set of rules $R$ for each language constructs.
In particular, in \STLC{}, $S_p=\{\Exp,\Val,\Type\}$.
The evaluation rules of \STLC{} are defined in Fig. \ref{fig:stlc} whose output sort is $\Val$.

\begin{figure}[t!]
  \begin{align*}
    \ctr{True}{\<true>}    & \cqq \<true> \\
    \ctr{False}{\<false>}  & \cqq \<false> \\
    \ctr{If}{\<if>~e_1~e_2~e_3} & \cqq \HH{e_1}: \branch{
      \<true>  |> \HH{e_2} \\&
      \<false> |> \HH{e_3}
    } \\
    \ctr{Abs}{λx\!:\!t.~e} & \cqq λx\!:\!t.~e \\
    \ctr{App}{e_1~e_2} & \cqq \lt~λx\!:\!t.e=\HH{e_1};~\lt~v_2=\HH{e_2};~\HH{e[v_2/x]}
  \end{align*}
  \caption{Evaluation Rules of \STLC}
  \label{fig:stlc}
\end{figure}

\begin{requirement}
  For any constructor $c$ with sort $\Exp$, the evaluation rule should be unique.
  % (Variable has no evaluation rules.)
\end{requirement}

When a closed expression $\cexp$ is evaluated in $L$,
 we need to find a rule in $L$ and apply it to $\cexp$.
We write $L[\cexp]=Σ(b)$, if there exists a rule $\textsc{Name}(c~\mexp_{x_1}\cdots \mexp_{x_n}) \cqq b$ in $R$ and an environment $Σ$,
 so that $\cexp=Σ(c~\mexp_{x_1}\cdots \mexp_{x_n})$.
% If a constructor $c\in C$ can be used to build a value in $L$,
%  $c$ is called a value constructor. 

\subsection{Semantics of Meta-Language}

In this section, we present the semantics of our meta-language.
% We pick the big-step operational semantics.
Considering that the interpretation of our meta-language is actually a concrete interpretation,
 the semantics of our meta-language can be covered by Skeleton.
But for clarity, we follow the style of modular structural operational semantics (MSOS) \cite{msos}.
The semantics of meta-language can be defined by the relation $\cb \rr{X} \mexp$,
 where $\cb$ is a closed bone, and $\mexp$ is the result.
$X$ is a label, to describe transformation of global state, related to monad $m$.
For example, let $m$ be $\mathit{State~Store}$. 
Then $X=\langle σ,σ'\rangle$, where $σ$ is the current store, and $σ'$ is the updated store.
We will abbreviate $\cb \rr{X} \mexp$ to $\cb \lr \mexp$ if $X$ is identity morphism, i.e., $σ=σ'$.
Label composition $X_1;X_2$ is used to compose the labels of computation of sub-bones.
We require that $X_1$ and $X_2$ are composable: if $X_1=\langle σ,σ'\rangle$ and $X_2=\langle σ',σ''\rangle$,
then $X_1;X_2=\langle σ,σ''\rangle$. 
The semantics of meta-language are given in Fig. \ref{fig:seman-meta}.
And the evaluation rules of filters are pre-defined. 
For example, the rule of $\<store>$ can be defined as follows:
\[ \inference{σ'=\{σ,l\mapsto \mexp\}}{store(l,\mexp) \rr{\langleσ,σ'\rangle} ()}. \]
 
\begin{figure}
  \begin{gather*}
    \inference[Expression]{}{\mexp \lr \mexp} \\
    % \inference[PureFun]{\mexp'_1 \lr v_1 ~\cdots~ \mexp'_n \lr v_n & v=f_p(v_1\cdots v_n)}{f_p(\mexp'_1\cdots \mexp'_n) \lr v} \\
    \inference[Hook]{L[\mexp] \rr{X} v}{H(\mexp) \rr{X} v} \\
    % \inference[Filter]{\mexp_1 \lr v_1 ~\cdots~ \mexp_n \lr v_n & F(v_1\cdots v_n) \rr{X} v}{F(\mexp_1\cdots \mexp_n) \rr{X} v} \\
    \inference[LetIn]{b_1 \rr{X_1} v_1 & Σ=\mathit{match}\ (v_1,pat) & Σ(b_2) \rr{X_2} v_2}{\Let{pat}{b_1} b_2 \rr{X_1;X_2} v_2} \\
    \inference[Branch]{b\rr{X_1} v & (br_i=pat_i|>b_i)_{i\in[1..n]} &
      \text{FAIL}=\mathit{match}(v,pat_j)_{j\in [1..k-1]} \\ 
      Σ=\mathit{match}(v,pat_k) & Σ(b_k) \rr{X_2} v_k}
      {b:(br_1\cdots br_n) \rr{X_1;X_2} v_k}
  \end{gather*}
  \caption{Semantics of Meta-Language}
  \label{fig:seman-meta}
\end{figure}

%%%%%%%%%%%%%%%%%%%%%%%%%%%%%%%%%%%%%%%%%%%%%
\subsection{Host Languages with Structural Semantics}\label{sec:m-host}

Remember that our host language is designed to define DSL.
In overview, we have given an example of evaluation rule derivation.
The abstraction is guaranteed because the evaluation rule of $\<if>$ is applied.
We find that if we can always apply corresponding evaluation rules to the evaluations containing language constructs.
And when a language construct appears but is not used for evaluation, that language construct will be kept in the rule, which breaks abstraction.

We call this \textit{structural} property.
% An evaluation rule is \textit{structural} if it guarantees that all subexpressions are used after direct evaluation and evaluated expressions are subexpressions.
The structural here is to be distinguished from the structural operation semantics \cite{sos}.
 Our structural refers to a bottom-up computation, similar to a $\mathit{fold}$ or catamorphism over the syntax.
If some subexpression $e_i$ is used without evaluation directly, 
 or some expression being evaluated is not $e_i$,
 then the rule is not structural.
Most language constructs can be written in a structural way.
In the case of \STLC, the evaluation rule of $\<if>$ is structural.
And the evaluation rules of lambda abstraction and application are not structural.
The former uses the subexpression not evaluated, while the latter evaluates an expression containing substitution (used as an argument of substitution).
We rewrite these rules with nonstructural parts underlined:
% We rewrite these rules in the style of meta-language, where nonstructural parts are highlighted:
% weak alert
% \newcommand{\wkalt}[1]{\colorbox{lightgray}{#1}}
\newcommand{\wkalt}[1]{\textcolor{magenta}{#1}}
\begin{align*}
  \ctr{Abs}{λx:t.~e} & \cqq λx:t.~\underline{e} \\
  \ctr{App}{e_1~e_2} & \cqq \Let{λx:t.e}{\HH{e_1}} \Let{v_2}{\HH{e_2}} \HH{\underline{e[v_2/x]}}.
\end{align*}

\begin{definition}[Structural Evaluation Rule]\label{def:str}
A rule $\textsc{Name}(c~\mexp_{x_1}\cdots \mexp_{x_n})=b$ is structural, if
\begin{itemize}
  \item for each $\mexp_{x_i}$ satisfying $sort(\mexp_{x_i})=sort(c)$, 
    $\mexp_{x_i}$ is not used to be matched, as an argument of any filter or substitution, or applied by any constructor in $b$;
  \item all the expressions that are hooked $H(\mexp)$ in $b$ must be $\mexp_{x_i}$.
  \item all the constructor in $b$ must be used to construct a value.
\end{itemize}
\end{definition}

Structural evaluation rules are the key to ensure the abstraction in semantics lifting.
In other words, any language construct with nonstructural evaluation rule needs to be carefully accounted for in terms of how it will affect the semantics lifting.
% The typing rules, fortunately, are all structural in \STLC.

In addition, pattern matching should be a match of a value of computation,
 i.e. the following requirement should be satisfied.

\begin{requirement}\label{req:pat}
  The bone used for pattern matching must be of sort $\Val$ or a base sort.
\end{requirement}

% This rule is a weaker version of the structural rule,
%  which requires that parameters $\mexp_{x_i}$ cannot be used for pattern matching directly.

% To formalize definition \ref{def:str} and requirement \ref{req:pat}, we define $\mathit{checker}$:
