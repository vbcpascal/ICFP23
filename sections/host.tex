\section{Meta-Language and Host Languages}

% 介绍元语言的语法和语义,和生成到Haskell的翻译过程
% 宿主语言需要说我们要的是模块化的语义,即

In this section, we will discuss the syntax and semantic of our meta-language.

\subsection{Meta-Language}

As mentioned above, our meta-language is inspired by Skeleton.
But it differs in the problems and is therefore adapted.
Skeleton focuses on illustrating the consistency of concrete and abstract interpretations.
The meaning of filters (meta-operations in Skeleton) varies in different goals.
We concentrate on evaluation and typing.
As an interpretation of Skeleton, it is intuitively close to Haskell.
For example, instead of branches, we use pattern matching;
 instead of defining the merging of the interpretation of branches,
 we will pick the first branch that successfully matches.
In addition, we support monad at the meta-language level.
We will show the syntax of the meta-language, 
 and how the meta-language is translated to Haskell to illustrate its semantics.

\subsubsection{Expressions of Meta-Language}
Various \textit{expressions} in the meta-language are used to encode the language constructs in the host language or the DSL, such as expressions and types.
(We use $L$ to refer to a language defined by meta-language like \STLC.
In the remainder of this section, we will indicate such language with an $L$ to distinguish from the meta-language.
Also, expressions $\mexp$ in meta-language need to be sperated from expressions in language $L$.)

An expression $\mexp$ is built by \textit{base expressions}, like literals and identifiers, ranged by $\mexp_b$;
 \textit{expression variables}, ranged by $\mexp_x$;
 and \textit{compound expressions} by \textit{constructors}, ranged by $c$.
Each language construct in $L$ corresponds to a constructor $c$ in meta-language.
To distinguish between various sorts of language constructs,
 we define $S$ for a finite set of \textit{sorts}, ranged over by $s$.
A sort $s$ can be either an \textit{base sort} like $\Int$, $\Id$;
 or a \textit{program sort} like $\Exp$, $\Type$ in \STLC.
The sort of an expression $\mexp$, written as $sort(\mexp)=s$,
 illustrate its role in language $L$.
A constructor $c$ has a signature $sig(c)$, which is of the form $(s_1\cdots s_n)->s$,
 where $s_1\cdots s_n$ are appropriate sorts of parameters of $c$,
 and $s$ is the sort of expressions constructed by $c$, briefly stated as sort of $c$.
A compound expression, written as $c(\mexp_1,\cdots,\mexp_n)$ or $c~\mexp_1\cdots\mexp_n$,
 is legal, if $sort(\mexp_i)=s_i$ for every $i\in [1,n]$.

\begin{example}
$λx\!:\<bool>.~e$ in language $L$, is constructor $\<Abs>$ applied to ``x'', $\<bool>$ and expression variable $e$.
 ``x'' are base expressions, and $sort(``x")=\Id$;
 $\<bool>$ is constructor $\<Bool>$ applied to nothing.
The signature of $\<Abs>$ is $sig(\<Abs>)=(\Id,\Type,\Exp)->\Exp$, and its sort is $\Exp$.
The signature and sort of $\<Bool>$ are both $\Type$.
\[
  \begin{array}{ccc}
    \text{Language } L & & \text{Meta-Language} \\
    λx\!:\<bool>.~e & \leftrightarrow & \<Abs>(``x",\<Bool>(),e) \text{ or } \<Abs>~``x"~\<Bool>~e
  \end{array}
\]

\end{example}

We use $i$ for expressions with sort $\Int$, $x$ for $\Id$, $e$ for $\Exp$, and $t$ for $\Type$.
A \textit{closed expression} $\cexp$ is an expression with no expression variables, generally refering to a concrete component in language $L$.
Let $Σ$ be an \textit{environment} mapping from expression variables to expressions. 
We write $Σ(\mexp)$ to replace all expression variables $\mexp_x\in dom(Σ)$ with $Σ(\mexp_x)$ in $\mexp$.
An environment $Σ$ is \textit{closed} for expression $\mexp$, when $Σ(\mexp)$ is a closed expression.

\subsubsection{Rules of Meta-Language}

\begin{figure}
  \begin{align*}
    d \in Body 
      & ::= b  \\
      & \quad \mid b:(br_1\cdots br_n) & \text{(Pattern Matching)} \\
      & \quad \mid \Let{pat}{b_1} b_2 & \text{(Let Binding)} \\
    b \in Bone
      & ::= exp' & \text{(General Expression)} \\
      & \quad \mid h(exp') & \text{(Recursive Computation)} \\
      % & \quad \mid H(e_1[e_2/x]) & \text{(Recursive Computation with Substitution)} \\
      & \quad \mid f_m(exp'_1\cdots exp'_n) & \text{(Monadic Meta-function)} \\
    exp' \in \Exp_G
      & ::= exp & \text{(Expression)} \\
      & \quad \mid f_p(exp'_1\cdots exp'_n) & \text{(Pure Meta-function)} \\
    br \in Branch
      & ::= pat |> b \\
      & \quad \mid exp' |> b \\
    pat \in \mathit{Pattern}
      & ::= exp_x & \text{(Expression Variable)} \\
      & \quad \mid c~pat_1\cdots pat_n & \text{(Compound Pattern)}
  \end{align*}
\caption{Body of Rules in Meta-Language}
\label{fig:body}
\end{figure}

A rule of meta-language has the shape $h(c~\mexp_{x_1}\cdots \mexp_{x_n}) \cqq d$, 
 where $h$ is the name of interpretation like $\mathcal{E}$ and $\mathcal{T}$,
 $c$ is a constructor, 
 $\mexp_{x_1}\cdots \mexp_{x_n}$ are expression variables,
 and $d$ is the rule body.
An interpretation $h$ has a signature $sig(h)=s_1->m~s_2$, 
 where $s_1$ states for which $h$ computes,
 and $s_2$ states for the result of $h$,
 $m$ is the monad for side-effects.
When no side effects are introduced, $m$ can be omitted.
A body is composed of bones, shown in Fig. \ref{fig:body}.
A bone, which represents an operation, 
 can be either a general expression, 
 a recursive computation,
 or a meta-function.

General expressions are expressions $\mexp$ or computation of expressions with $f_p$.
Two types of meta-functions are mentioned here:
 monadic meta-functions, ranged over by $f_m$,
 pure meta-functions, ranged over by $f_p$.
% They are the same if computation is pure.
% But when monad is introduced, 
When monad is introduced,
 $f_m$ is function containing side effects, 
  whose signature is like $sig(f_m)=(s_1\cdots s_n)->m~s$;
 while $f_p$ is not,
  whose signature is of the form $sig(f_p)=(s_1\cdots s_n)->s$.
Most of these functions are pre-defined.
For example, $searchCtx$, $updateCtx$ and $restoreCtx$ are monadic meta-functions,
 equality checking $(=)$ and substitution are pure meta-functions.
Interestingly, as the result, the use of monad in the body of rules is implicit,
 where $return$ or $bind$ need not be written explicitly.

\subsubsection{Language $L$ Defined in Meta-Language}
A \textit{Language} $L$ consists of a set of program sorts $S_p$;
 a set of language constructs $C$ with signatures;
 a set of interpretations $H$ with rules;
 a set of meta-functions $F$, seprerated by monadic ones $F_m$ and pure ones $F_p$.
In particular, in \STLC and its extensions, $S_p=\{\Exp,\Type\}$.
$H$ can be seperated into a set of evaluation rules $R_{\mE}$ and a set of typing rules $R_{\mT}$,
 where $sig(\mE)=\Exp->m_1~\mE$ and $sig(\mT)=\Exp->m_2~\mT$.
Substitution, as a pure meta-function, is divided from $F_p$ and described as a set of substitution rules $R_{subst}$,
 whose definition is defined by develops.
\[ L=\langle S_p, C, R_{\mE}, R_{\mT}, R_{subst}, F_p, F_m \rangle \]
% Fig. shows an example of STLC definition in our meta-language.

% \begin{figure}
%   \begin{align*}
%     S_p & = \{ \Exp, \Type \} \\
%     H & = \{ \mathcal{E}: \Exp->\Exp, \mathcal{T}: \Exp->State~CtxStack~\Type \} \\
%     F_m & = \{ searchCtx, updateCtx, restoreCtx \} \\
%     F_p & = \{ (=), subst \}
%   \end{align*}
% \end{figure}

\subsubsection{Code Generation.}


\subsection{Host Languages}

Our host language is designed to define DSL.
For the purpose of semantic derivation, 
 each language constructs of the host language should have a structural computation.
 the evaluation of an expression $c~e_1\cdots e_n: \Exp$ should be an operation of the evaluation of each subexpression $e_i$ if $e_i:\Exp$.
If some $e_i$ is used without evaluation, 
 or some expression being computed is not $e_i$,
 then the rule is not structural.
The evaluation rules of $\<if>$ and $\<and>$ are structural.
Structural evaluation rules are the key to ensure the abstraction in semantic derivation.
In other words, any language construct with unstructural evaluation rule needs to be carefully accounted for in terms of how it will affect the semantic derivation.

In the case of STLC, the evaluation rules of lambda abstraction and application are not structural.
The former uses the $e_i$ not evaluated, 
 while the latter evaluates an expression containing substitution.
We rewrite these two rules according to the meta-language definition above as follows, 
 where $subst \in F_p$:
% weak alert
\newcommand{\wkalt}[1]{\colorbox{lightgray}{#1}}
\begin{align*}
  \EE{\<Abs>~x~t~e} & \cqq \<Abs>~x~t~\wkalt{$e$} \\
  \EE{\<App>~e_1~e_2} & \cqq \Let{\<Abs>~x~t~e}{\EE{e_1}} \Let{v_2}{\EE{e_2}} \wkalt{$\EE{subst~e~x~v_2}$}.
\end{align*}

\todo{Safety? cannot be expressed by big-step semantics}


