\section{Host Language Extension}\label{sec:ex}

In this section, we discuss how to extend the host language when it is not expressive enough.
The extensions can be seperated into two types:
 meta-extensions and monadic extensions.
The former adds new vocabularies to the host language,
 for example adding integers to \STLC.
Meta-language extensions are similar to defining a host language.
The latter is adding new features to the meta-language by monad like reference.
The original language definitions can be kept intact when introducing new language features.

\subsection{Meta-Extensions}

Meta-extensions are direct extensions to the host language by
 adding new constructors $C_n$ with corresponding rules, and new filters $F_n$.
% This expansion has the following requirements: for each $c \in C_n$,
%  (1) $sort(c) \in S_p$, i.e. $\Exp$ or $\Type$;
%  (2) if $sort(c)=\Exp$, then the evaluation, typing and substitution rules should be specified.

\begin{example}
  Consider integers are expected to be supported via literals.
  % New constructors, $\<Lit>$, $\<Plus>$, $\<Lt>$ and $\<Eq>$ as $\Exp$, $\<Int>$ as $\Type$, will be added.
  Fig. \ref{fig:int} shows the formal definition of integer extension. 
  % where $F_n=\{(=),(+),(<)\}$, all of them are pure meta-functions.
\end{example}

\begin{figure}
  \begin{align*}
    & \ctr{Lit}{lit~i} \cqq lit~i \\
    & \ctr{Plus}{e_1+e_2} \cqq \Let{lit~i_1}{\EE{e_1}} \Let{lit~i_2}{\EE{e_2}} plus(lit~i_1,lit~i_2) \\
    & \ctr{Lt}{e_1 < e_2} \cqq \Let{lit~i_1}{\EE{e_1}} \EE{e_2}:\branch{
        lit~i_2 \mid i_1 <_i i_2 |> \<true> \\&
        lit~\_ |> \<false> 
      } \\
    & \ctr{Eq}{e_1=e_2} \cqq \Let{lit~i_1}{\EE{e_1}} \EE{e_2}:\branch{
        lit~i_2 \mid i_1 =_i i_2 |> \<true> \\&
        lit~\_ |> \<false> 
      }
  \end{align*}
  \caption{Formal Definition of Integer Extension}
  \label{fig:int}
\end{figure}

% \begin{figure}
%   \begin{align*}
%     & \<Int>:\Type \\ 
%     & \<Lit>:\Int->\Exp & 
%     & \EE{\<Lit>~i} \cqq \<Lit>~i \hspace{5em} \TT{\<Lit>~i} \cqq \<Int> \\
%     & & & (\<Lit>~i)[e/x] = \<Lit>~i \\
%     & \<Plus>:(\Exp,\Exp)->\Exp &
%     & \EE{\<Plus>~e_1~e_2} \cqq \Let{\<Lit>~i_1}{\EE{e_1}} \Let{\<Lit>~i_2}{\EE{e_2}} \<Lit>~(i_1+i_2) \\
%     & & & \TT{\<Plus>~e_1~e_2} \cqq \Let{\<Int>}{\TT{e_1}} \Let{\<Int>}{\TT{e_2}} \<Int> \\
%     & & & (\<Plus>~e_1~e_2)[e/x]=\<Plus>~e_1[e/x]~e_2[e/x] \\
%     & \<Lt>:(\Exp,\Exp)->\Exp &
%     & \EE{\<Lt>~e_1~e_2} \cqq \Let{\<Lit>~i_1}{\EE{e_1}} \EE{e_2}:\branch{
%         \<Lit>~i_2 \mid i_1 < i_2 |> \<True> \\&
%         \<Lit>~\_ |> \<False> 
%       } \\
%     & & & \TT{\<Lt>~e_1~e_2} \cqq \Let{\<Int>}{\TT{e_1}} \Let{\<Int>}{\TT{e_2}} \<Bool> \\
%     & & & (\<Lt>~e_1~e_2)[e/x]=\<Lt>~e_1[e/x]~e_2[e/x] \\
%     & \<Eq>:(\Exp,\Exp)->\Exp &
%     & \EE{\<Eq>~e_1~e_2} \cqq \Let{\<Lit>~i_1}{\EE{e_1}} \EE{e_2}:\branch{
%         \<Lit>~i_2 \mid i_1 = i_2 |> \<True> \\&
%         \<Lit>~\_ |> \<False> 
%       } \\
%     & & & \TT{\<Eq>~e_1~e_2} \cqq \Let{\<Int>}{\TT{e_1}} \Let{\<Int>}{\TT{e_2}} \<Bool> \\
%     & & & (\<Eq>~e_1~e_2)[e/x]=\<Eq>~e_1[e/x]~e_2[e/x] \\
%   \end{align*}
%   \caption{Formal Definition of Integer Extension}
%   \label{fig:int}
% \end{figure}

\subsection{Monad Extensions}\label{sec:dsl-monad}

Monad extension \cite{monad-1,monad-2} is the technique for side effects in computation without changing the original semantic definition.
By monad $m$, various language features can be defined, such as environment, store, nondeterminism, I/O etc.
In order to put these together in a modular way, we use a mechanism called \textit{monad transformers} \cite{monad-tr}, ranged over by $m_t$.
We will add the language features in \STLC,
 in order to illustrate that when monad is extended,
  the user only needs to focus on the new language features,
  the original rules remain unchanged,
  and the original filters will be automatically lifted.
As we have already said,
 in the pure evaluation,
 we assume that there is an Identity monad.
This is for the sake of consistency in code generation on the one hand,
 and also allows the user to introduce a new monad without having to distinguish whether it is added as a monad or a monad transformer.
In addition, when users need IO support, they can change the internal Identity monad to IO monad will not lead to the destruction of the outer monad.

We start with introduce reference to \STLC, named \textsc{Ref}.

\subsubsection{Evaluation with Store}

Monad extensions are often accompanied by meta-extensions.
The basic language constructs of reference are allocation $ref~e$, dereferencing $!e$ and assignment $e_1:=e_2$.
To carry through the modification on store, 
 the out sort of the sorts changes from $\Identity~\Val$,
 to $(\mathit{StateT~Store~Identity})~\Val$, abbreviated as $M_e~\Val$,
 where $\mathit{Store}$ maps locations to values.

An abstract index of store is a location (used as a base sort $Location$).
A reference will be evaluated to a location expression.
A location in \textsc{Ref} is written as $l$ directly.
The value for location is constructed by $loc$ whose sort is $Location->\Val$.
New (monadic) filters are also necessary for evaluation rules:
\begin{itemize}
  \item $\newLoc:M_e~Location$ returns a newly allocated location;
  \item $\store:(Location,\Val)->M_e~()$ updates the store to make the location contain the value;
  \item $\fetch:Location->M_e~\Val$ gets the value at the location from the store.
\end{itemize}
Given these filters, Fig. \ref{fig:ref_eval} shows the evaluation rules of these newly defined language constructs.

\begin{figure}
  \begin{align*}
    \ctr{Loc}{loc~l} & \cqq loc~l \\
    \ctr{Ref}{ref~e} & \cqq \Let{v}{\EE{e}} \Let{l}{\newLoc} \Let{()}{\store~(l,v)} loc~l \\
    \ctr{Deref}{!e} & \cqq \Let{loc~l}{\EE{e}} \fetch~l \\
    \ctr{Ass}{e_1:=e_2} & \cqq \Let{loc~l}{\EE{e}} \Let{v_2}{\EE{e_2}} \Let{()}{\store~(l,v)} ()
  \end{align*}
  \caption{Evaluation Rules of \textsc{Ref}}
  \label{fig:ref_eval}
\end{figure}

When introducing new monad, filters $F:(s_1\cdots s_n)->\Identity~s$ should be lifted to $\lifted{F}:(s_1\cdots s_n)->M_e~s$
 by $\mathit{lift}$ operation.
\[ \lifted{F}(\mexp_1\cdots \mexp_n) = \mathit{lift}~(F(\mexp_1\cdots \mexp_n)). \]

\subsubsection{IO Monad}

Unlike other monads, IO monad cannot be used as a monad transformer,
 which can be the base monad only.
For this, we always put an identity monad or IO monad at bottom,
 with other monads layered on top.
When I/O is required, replace identity monad with IO monad.
Then lift I/O functions (filters), with other filters unchanged.

Monad transformers allow us touch high-level language features, but skill keep the access to low-level details.
For example, in \textsc{Ref}, out flow sort of rules is $\mathit{StateT~Store~Identity}~\Val$
To support I/O, the sort becomes $\mathit{StateT~Store~IO}~\Val$.
And meta-functions provided by IO monad are lifted once:
\[ \lifted{\<print>}~e = \mathit{lift}~(\<print>~e). \]
