\section{Language Lifting}

Language lifting is the method for generating a stondalone DSL from an extended host language and translation rules.
Ignoring the modification of the syntax, we will present how to derive computation rules and type rules.
In particular, we will discuss language lifting for those whose host language is STLC or extended STLC.

\subsection{Preprocessing: Lambda Lifting}

We have already exemplified how lambda abstractions break the abstraction property in semantic derivation.
The solution to this problem is to expose the lambda abstractions in the translation rules 
 and define them as new language constructs of the DSL.
For the outermost lambda abstractions, the transformation is straightforward. 
The inner lambda abstractions, on the other hand, need to be extracted recursively.
The challenge here is to ensure that the translation rules are closed.
Formally, for the translation rule $tr$ which defines $c_n'$:
\newcommand{\laml}[1]{\uparrow_λ\!\!(#1)}
\begin{align}
  \laml{(λx:t.e_1)~e_2} & = (λx:t.\laml{e_1})~\laml{e_2} \label{ll:let} \\
  \laml{λx:t.e} & = λx:t.(c_n'~\exp_1..\exp_p ~ x_1..x_q) \\[-3pt]
    & \text{generating } c_n'~\exp_1..\exp_p ~ e_1..e_q => \laml{e[e_1/x_1..e_q/x_q]} \\[-3pt]
    & \text{for each expression variable } \exp_i \text{ in } e \text{ and variable } x_j \text{ not bound in } e \\
  \laml{c~exp_1\cdots exp_n} & = c~\laml{exp_1}\cdots\laml{exp_n} \\
  \laml{exp_x} & = exp_x \\
  \laml{exp_b} & = exp_b
\end{align}
where $\laml{\bullet}$ is lambda lifting transformation, with the generation of some translation rules.
The first rule states that if a lambda abstraction is applied directly in the translation rule, 
 then it does not need to be lifted because it does not break the abstraction (which can essentially be written as a $\<let>$).
The second is the main rule, which generates a new constructor $c_n'$ to describe the semantics of a lambda abstraction.
Any expression variables of $LHS(tr)$ used in $e$ (i.e., parameters of $c_n$) need to be applied by $c_n'$.
In addition, any variable $x_j$ which have been bound and captured by $e$ are also required to be included,
 by an new expression variable $e_j$, which ensures the closure of translation rules.
Note that the newly generated $c_n'$ needs to process lambda lifting recursively,
 and may generate some more translation rules.

\todo{examples}

\begin{lemma}
  Lambda lifting preserves requirements \ref{req:no-recursion} and \ref{req:close}.
\end{lemma}

\subsection{Semantic Derivation}

In order for the DSL to support new language constructs defined by translation rules, 
 it is necessary to provide their evaluation rules and type rules.
In this section, a generic algorithm will be presented, 
 to illustrate how to derive the rules for translation rules.
For a DSL defined by multiple translation rules, 
 we derive the evaluation and typing rules and add them to the language individually.

% And now we will give the algorithm of semantic derivation. 
Suppose a host language based on STLC is defined as $L=\langle S_p, C, R_{\mE}, R_{\mT}, R_{subst}, F\rangle$,
where $S_p=\{\Exp,\Type\}$,
and a translation rule $tr$ has shape $c_n~exp_1\cdots exp_n => e$,
where $sort(c_n)=\Exp$ and $tr$ satisfies the requirements.
Let $L'=\langle S_p, C', R'_{\mE}, R'_{\mT}, R'_{subst}, F\rangle$ be the new language that supports $c_n$ (i.e., $C'=C \cup \{c_n\}$).

\subsubsection{Algorithm}

In overview, the core idea of the semantic derivation has been shown: 
 expand expressions evaluation recursively according to the rules in $L$ 
 until all the evaluation are unexpandable.
Formally, $\EE{LHS(tr)}=\dd(\EE{RSH(tr)})$, the definition of $\dd$ is shown in Fig. \ref{fig:sd}.

\begin{figure}
  $\EE{LHS(tr)}=\dd(\EE{RSH(tr)})$
  \begin{align}
    \noalign{\raggedright \hspace{4em} Body:}
    \dd(b:(br_1\cdots br_n)) & = \dd(b):(\dd(br_1)\cdots \dd(br_n)) \\
    \dd(\Let{pat}{b_1} b_2) & = \Let{pat}{\dd(b_1)} \dd(b_2) \\
    \noalign{\raggedright \hspace{4em} Bone:}
    \dd(\EE{exp_x}) & = \EE{exp_x} \\  
    \dd(\EE{exp_b}) & = \EE{exp_b} \\
    \dd(\EE{c~exp_1\cdots exp_m}) 
      & = \dd(Σ(d)) \hspace{20pt} \text{if there exists $Σ$ and $\EE{exp}\cqq d$,} \\
      & \hspace{61pt} \text{such that $Σ(exp)=c~exp_1\cdots exp_m$} \nonumber \\
    \dd(\EE{f_p~(exp_1'\cdots exp_n')}) & = f_p~(exp_1'\cdots exp_n') \\
    \dd(f_m(exp_1'\cdots exp_n')) & = f_m(exp_1'\cdots exp_n') \\
    \noalign{\raggedright \hspace{4em} General Expression:}
    \dd(exp') & = exp' \\
    \noalign{\raggedright \hspace{4em} Branch:}
    \dd(pat |> b) & = pat |> \dd(b) \\ 
    \dd(pat \mid exp' |> b) & = pat \mid exp' |> \dd(b)
  \end{align}
  \caption{Semantic Derivation}
  \label{fig:sd}
\end{figure}

To illustrate the algorithm, some examples will be presented.
Example \ref{exm:nand} defines $\<nand>$ by other translation rules $\<not>$ and $\<and>$.
Assume that the semantics of $\<not>$ and $\<and>$ have been derived earlier.
We will observe that how $\dd$ works recursively.
For clarity, the parts that need to be applied by $\dd$ are underlined.

\begin{example}\label{exm:nand}
  $e_1~\<nand>~e_2 => \<not>~(e_1~\<and>~e_2)$:
  \begin{align*}
       & \EE{e_1~\<nand>~e_2} \\
    =~ & \dhl{\EE{\<not>~(e_1~\<and>~e_2)}} \\
    =~ & \dhl{\EE{e_1~\<and>~e_2}
            :\branch{\<true>|>\<false> \\& \<false>|>\<true>}} \\
    =~ & \dhl{\EE{e_1~\<and>~e_2}}
            :\branch{\<true>|>\dhl{\<false>} \\& \<false>|>\dhl{\<true>}} \\
    =~ & \dhl{\EE{e_1}
            :\branch{\<true>|>\EE{e_2} \\& \<false>|>\<false>}
           }:\branch{\<true>|>\<false> \\& \<false>|>\<true>} \\
    =~ & \EE{e_1}
          :\branch{\<true>|>\EE{e_2} \\& \<false>|>\<false>}
          :\branch{\<true>|>\<false> \\& \<false>|>\<true>} 
  \end{align*}  
\end{example}

The language construct $\<let>$ is a common syntactic sugar, defined by lambda abstraction and application.
Considering $\<let>$ as a translation rule, example \ref{exm:let} presents the derivation of its evaluation rule.
Note that the RHS of the translation rule satisfies \ref{ll:let}, and no language constructs will be generated in lambda lifting.
The last step is simplification, since that both left and right sides are constructed by $\<Abs>$ (i.e., $λ$ in language).
simplification is sometimes useful for abstraction property, for instance in this example,
 where the evaluation rule of $\<let>$ no longer depends on lambda abstraction (even if this is allowed).
In addition, substituion, as a meta-function, appears in the following derivation, which does not affect the expansion.
In \todo{}, we will discuss substituion more deeply.

\begin{example}
  $\<let>~x:t=e_1~\<in>~e_2 => (λx:t.e_2)~e_1$:
  \begin{align*}
       & \EE{\<let>~x:t=e_1~\<in>~e_2} \\
    =~ & \dhl{\EE{(λx:t.e_2)~e_1}}  \\
    =~ & \dhl{\Let{λx':t'.e}{\EE{λx:t.e_2}} \Let{v_1}{\EE{e_1}} \EE{e[v_1/x']}} \\
    =~ & \Let{λx':t'.e}{\dhl{\EE{λx:t.e_2}}} \dhl{\Let{v_1}{\EE{e_1}} \EE{e[v_1/x']}} \\
    =~ & \Let{λx':t'.e}{λx:t.e_2} \Let{v_1}{\dhl{\EE{e_1}}} \dhl{\EE{e[v_1/x']}} \\
    =~ & \Let{λx':t'.e}{λx:t.e_2} \Let{v_1}{\EE{e_1}} \EE{e[v_1/x']} \\
    =~ & \Let{x'}{x} \Let{t'}{t} \Let{e}{e_2} \Let{v_1}{\EE{e_1}} \EE{e[v_1/x']}
  \end{align*}
  \label{exm:let}
\end{example}


Correctness is that, given a closed expression $e:\Exp$ in $L'$, the value by evaluation should be the same as the value,
of fully translating first and then evaluating in $L$.
We use subscripts to distinguish in which language the computation is performed.
Then:
\begin{align*}
  \EELL{e} & = \EEL{\DS(e)} \\
  \TTLL{e} & = \TTL{\DS(e)}
\end{align*}
If $e$ is constructed by $c_n$, then further, the following equation holds:
\begin{align*}
  \EELL{e} & = \EELL{\ds(e)} \\
  \TTLL{e} & = \TTLL{\ds(e)}
\end{align*}

