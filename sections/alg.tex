\section{Language Lifting}\label{sec:alg}

Language lifting is the method for generating a stondalone DSL from an extended host language and translation rules.
Ignoring the modification of the syntax, we will present how to derive computation rules and type rules.
In particular, we will discuss language lifting for those whose host language is \STLC{} or extended \STLC.

\subsection{Preprocessing: Lambda Lifting}

We have already exemplified how lambda abstractions break the abstraction property in semantic derivation.
The solution to this problem is to expose the lambda abstractions in the translation rules 
 and define them as new language constructs of the DSL.
For the outermost lambda abstractions, the transformation is straightforward. 
The inner lambda abstractions, on the other hand, need to be extracted recursively.
The challenge here is to ensure that the translation rules are closed.
Formally, for the translation rule $tr$ which defines $c_n'$:
\newcommand{\laml}[1]{\uparrow_λ\!\!(#1)}
\begin{align}
  \laml{(λx:t.e_1)~e_2} & = (λx:t.\laml{e_1})~\laml{e_2} \label{ll:let} \\
  \laml{λx:t.e} & = λx:t.(c_n'~\exp_1..\exp_p ~ x_1..x_q) \\[-3pt]
    & \text{generating } c_n'~\exp_1..\exp_p ~ e_1..e_q => \laml{e[e_1/x_1..e_q/x_q]} \\[-3pt]
    & \text{for each expression variable } \exp_i \text{ in } e \text{ and variable } x_j \text{ not bound in } e \\
  \laml{c~exp_1\cdots exp_n} & = c~\laml{exp_1}\cdots\laml{exp_n} \\
  \laml{exp_x} & = exp_x \\
  \laml{exp_b} & = exp_b
\end{align}
where $\laml{\bullet}$ is lambda lifting transformation, with the generation of some translation rules.
The first rule states that if a lambda abstraction is applied directly in the translation rule, 
 then it does not need to be lifted because it does not break the abstraction (which can essentially be written as a $\<let>$).
The second is the main rule, which generates a new constructor $c_n'$ to describe the semantics of a lambda abstraction.
Any expression variables of $LHS(tr)$ used in $e$ (i.e., parameters of $c_n$) need to be applied by $c_n'$.
In addition, any variable $x_j$ which have been bound and captured by $e$ are also required to be included,
 by an new expression variable $e_j$, which ensures the closure of translation rules.
Note that the newly generated $c_n'$ needs to process lambda lifting recursively,
 and may generate some more translation rules.

\todo{examples}

\begin{lemma}
  Lambda lifting preserves requirements \ref{req:no-recursion} and \ref{req:close}.
\end{lemma}

\subsection{Semantic Derivation}

In order for the DSL to support new language constructs defined by translation rules, 
 it is necessary to provide their evaluation rules and type rules.
In this section, a generic algorithm will be presented, 
 to illustrate how to derive the rules for translation rules.
For a DSL defined by multiple translation rules, 
 we derive the evaluation and typing rules and add them to the language individually.

% And now we will give the algorithm of semantic derivation. 
Suppose a host language based on \STLC is defined as $L=\langle S_p, C, R_{\mE}, R_{\mT}, R_{subst}, F\rangle$,
where $S_p=\{\Exp,\Type\}$,
and a translation rule $tr$ has shape $c_n~exp_1\cdots exp_n => e$,
where $sort(c_n)=\Exp$ and $tr$ satisfies the requirements.
Let $L'=\langle S_p, C', R'_{\mE}, R'_{\mT}, R'_{subst}, F\rangle$ be the new language that supports $c_n$ (i.e., $C'=C \cup \{c_n\}$).

In overview, the core idea of the semantic derivation has been shown: 
 expand expressions evaluation recursively according to the rules in $L$ 
 until all the evaluation are unexpandable.
Formally, $\EE{LHS(tr)}=\dd(\EE{RSH(tr)})$, the definition of $\dd$ is shown in Fig. \ref{fig:sd}.

\begin{figure}
  $\EE{LHS(tr)}=\dd(\EE{RSH(tr)})$
  \begin{align}
    \noalign{\raggedright \hspace{2em} \bf Body:}
    \dd(b:(br_1\cdots br_n)) & = \dd(b):(\dd(br_1)\cdots \dd(br_n)) \\
    \dd(\Let{pat}{b_1} b_2) & = \Let{pat}{\dd(b_1)} \dd(b_2) \\
    \noalign{\raggedright \hspace{2em} \bf Bone:}
    \dd(\EE{exp_x}) & = \EE{exp_x} \\  
    \dd(\EE{exp_b}) & = \EE{exp_b} \\
    \dd(\EE{c~exp_1\cdots exp_m}) 
      & = \dd(Σ(d)) \hspace{20pt} \text{if there exists $Σ$ and $\EE{exp}\cqq d$,} \\
      & \hspace{61pt} \text{such that $Σ(exp)=c~exp_1\cdots exp_m$} \nonumber \\
    \dd(\EE{f_p~(exp_1'\cdots exp_n')}) & = f_p~(exp_1'\cdots exp_n') \\
    \dd(f_m(exp_1'\cdots exp_n')) & = \EE{f_m(exp_1'\cdots exp_n')} \\
    \noalign{\raggedright \hspace{2em} \bf General Expression:}
    \dd(exp') & = exp' \\
    \noalign{\raggedright \hspace{2em} \bf Branch:}
    \dd(pat |> b) & = pat |> \dd(b) \\ 
    \dd(pat \mid exp' |> b) & = pat \mid exp' |> \dd(b)
  \end{align}
  \caption{Semantic Derivation}
  \label{fig:sd}
\end{figure}

To illustrate the algorithm, some examples will be presented.
Example \ref{exm:nand} defines $\<nand>$ by other translation rules $\<not>$ and $\<and>$.
Assume that the semantics of $\<not>$ and $\<and>$ have been derived earlier.
We will observe that how $\dd$ works recursively.
For clarity, the parts that need to be applied by $\dd$ are underlined.

\begin{example}\label{exm:nand}
  $e_1~\<nand>~e_2 => \<not>~(e_1~\<and>~e_2)$:
  \begin{align*}
       & \EE{e_1~\<nand>~e_2} \\
    =~ & \dhl{\EE{\<not>~(e_1~\<and>~e_2)}} \\
    =~ & \dhl{\EE{e_1~\<and>~e_2}
            :\branch{\<true>|>\<false> \\& \<false>|>\<true>}} \\
    =~ & \dhl{\EE{e_1~\<and>~e_2}}
            :\branch{\<true>|>\dhl{\<false>} \\& \<false>|>\dhl{\<true>}} \\
    =~ & \dhl{\EE{e_1}
            :\branch{\<true>|>\EE{e_2} \\& \<false>|>\<false>}
           }:\branch{\<true>|>\<false> \\& \<false>|>\<true>} \\
    =~ & \EE{e_1}
          :\branch{\<true>|>\EE{e_2} \\& \<false>|>\<false>}
          :\branch{\<true>|>\<false> \\& \<false>|>\<true>} 
  \end{align*}  
\end{example}

The language construct $\<let>$ is a common syntactic sugar, defined by lambda abstraction and application.
Considering $\<let>$ as a translation rule, example \ref{exm:let} presents the derivation of its evaluation rule.
Note that the RHS of the translation rule satisfies \ref{ll:let}, and no language constructs will be generated in lambda lifting.
The last step is simplification, since that both left and right sides are constructed by $\<Abs>$ (i.e., $λ$ in language).
simplification is sometimes useful for abstraction property, for instance in this example,
 where the evaluation rule of $\<let>$ no longer depends on lambda abstraction (even if this is allowed).
In addition, substituion, as a meta-function, appears in the following derivation, which does not affect the expansion.
In \todo{}, we will discuss substituion more deeply.

\begin{example}
  $\<let>~x:t=e_1~\<in>~e_2 => (λx:t.e_2)~e_1$:
  \begin{align*}
       & \EE{\<let>~x:t=e_1~\<in>~e_2} \\
    =~ & \dhl{\EE{(λx:t.e_2)~e_1}}  \\
    =~ & \dhl{\Let{λx':t'.e}{\EE{λx:t.e_2}} \Let{v_1}{\EE{e_1}} \EE{e[v_1/x']}} \\
    =~ & \Let{λx':t'.e}{\dhl{\EE{λx:t.e_2}}} \dhl{\Let{v_1}{\EE{e_1}} \EE{e[v_1/x']}} \\
    =~ & \Let{λx':t'.e}{λx:t.e_2} \Let{v_1}{\dhl{\EE{e_1}}} \dhl{\EE{e[v_1/x']}} \\
    =~ & \Let{λx':t'.e}{λx:t.e_2} \Let{v_1}{\EE{e_1}} \EE{e[v_1/x']} \\
    =~ & \Let{x'}{x} \Let{t'}{t} \Let{e}{e_2} \Let{v_1}{\EE{e_1}} \EE{e[v_1/x']}
  \end{align*}
  \label{exm:let}
\end{example}

\subsubsection{Properties}

With the semantic derivation algorithm, the evaluation rules of $L'$ are generated.
We expect these rules to have the following properties:
\begin{itemize}
  \item Correctness: a closed expression $\ce$ in $L'$, should have the same value as 
         that of expression fully translating first and then evaluation in $L$.
  \item Abstraction: any constructor used in evaluation rules of $L'$ is either a value constructor,
         or a DSL constructor.
\end{itemize}
In this section, we will presents these properties formally.

We use subscripts to distinguish in which language the computation is performed.
% Then, the correctness is
% \[ \EELL{\ce} \rr{X} v ~\text{ iff }~ \EEL{\DS{\ce}} \rr{X} \DS{v}. \]
For the proof of the correctness theorem, we will illustrate $\dd$ 
To prove the above conclusion, we first illustrate that $\dd$ does not influence the result of a closed bone $b$.

\begin{lemma}
  Support $b$ is a closed bone $b$. Then:
  \[ b \rr{X} v ~\text{ iff }~ \dd(b) \rr{X} v \]
\end{lemma}

\begin{proof}
  By induction on a derivation of $b \rr{X} v$. 
  The \textsc{Base}, \textsc{PureFun} and \textsc{MonadicFun} cases are immediate, since $C(b)=b$.
  The other cases are shown as follows:

  Case \textsc{Rec}: let $b=\EE{e}$. We find that $\dd$ has different forms according to $e$.
  We consider each case separately:
  \begin{itemize}
    \item $e$ is a base expression, or a pure meta-function invocation: $C(b)=b$;
    \item $e=c~\cexp_1\cdots \cexp_n$: we have $\EE{e}=L[e]$.
      Furthermore, according to the definition of $\dd$, 
      we can find a \textit{closed} environment $Σ$ and a rule $\EE{\mexp}\cqq b_1$ in $L$, such that $Σ(\mexp) = e$, $\dd(\EE{e})=\dd(Σ(b_1))$.
      Since $e$ is a closed variable, $Σ(b_1)$ is a closed bone, and $Σ(b_1)=L[e]$.
      By the induction hypothesis, $L[e]\rr{X} v$ iff $\dd(L[e])\rr{X} v$.
      Then, $\EE{e}\rr{X} v$, iff $\dd(\EE{e})\rr{X} v$.
  \end{itemize}

  Case \textsc{LetIn}: let $b=\Let{pat}{b_1} b_2$.
    Then $b\rr{X} v$, if and only if the following judgements are satisfied:  
    \[ b_1\rr{X_1}v_1 \quad Σ=match(pat,v_1) \quad Σ(b_2)\rr{X_2}v_2 \quad X=X_1;X_2, \]
     where $Σ(b_2)$ is a closed bone.
    By the induction hypothesis, we have
    \[ b_1\rr{X_1}v_1 \miff C(b_1)\rr{X_1}v_1 ,\quad Σ(b_2)\rr{X_2}v_2 \miff C(Σ(b_2))\rr{X_2}v_2. \]
    Therefore, we can apply rule \textsc{LetIn} to conclude \todo{lemma: C Sigma}
    \[ \Let{pat}{C(b_1)} C(b_2) \rr{X} v \]
    whose left-hand side equals to $C(b)$.

  Case \textsc{Branch}: Similar.
\end{proof}

Before the proof of correctness, we will give a stronger lemma,
 of which correctness theorem is an instance.
The full translation is defined on the expressions now,
 which can be extended to the bone $b$,
 by fully translation all the expressions in it recursively.
We use $\DB{b}$ to express the full translation on the bone and omit its formal definition.

\begin{lemma}\label{lemma:b-db}
  Support $\cb$ is a closed bone. Then:
  \[ \cb \rr{X} v \quad\miff\quad \DB{\cb} \rr{X} \DS{v} \]
\end{lemma}

\begin{proof}
  By induction on a derivation of $\cb \rr{X} v$. 
  The \textsc{Base} case are obvious.

  Case \textsc{PureFun} and \textsc{MonadicFun}: by assumption \ref{asm:fun-ds}. 
  
  Case \textsc{Rec}: suppose that $b=\EE{e}$. There are three cases should be considered.
  \begin{itemize}    
    \item $e=c_n~\cexp_1\cdots \cexp_n$ where $c_n$ is a new constructor.
      Then \todo{}:
      \begin{align*}
        & \EE{c_n~\cexp_1\cdots \cexp_n} \rr{X} v \\
        \text{iff}~~ & L'[c_h~\mexp_1\cdots \mexp_n]\rr{X} v & \text{by the rule \textsc{Rec}} \\
        \text{iff}~~ & \DB{L'[c_h~\mexp_1\cdots \mexp_n]} \rr{X} \DS{v} & \text{inductive hypothesis} \\
        \text{iff}~~ & L'[\DB{c_h~\mexp_1\cdots \mexp_n}] \rr{X} \DS{v} & \text{\todo{}} \\
        \text{iff}~~ & \EE{\DB{c_h~\mexp_1\cdots \mexp_n}} \rr{X} \DS{v} & \text{by the rule \textsc{Rec}} \\
      \end{align*}
    \item $e=c_h~\mexp_1\cdots \mexp_n$ where $c_h$ is a host constructor.
      Then:
      \begin{align*}
        & \EE{c_h~\mexp_1\cdots \mexp_n} \rr{X} v \\
        \text{iff}~~ & L'[c_h~\mexp_1\cdots \mexp_n]\rr{X} v & \text{by the rule \textsc{Rec}} \\
        \text{iff}~~ & \DB{L'[c_h~\mexp_1\cdots \mexp_n]} \rr{X} \DS{v} & \text{inductive hypothesis} \\
        \text{iff}~~ & L'[\DB{c_h~\mexp_1\cdots \mexp_n}] \rr{X} \DS{v} & \text{\todo{}} \\
        \text{iff}~~ & \EE{\DB{c_h~\mexp_1\cdots \mexp_n}} \rr{X} \DS{v} & \text{by the rule \textsc{Rec}} \\
      \end{align*}
    \item $e=f_p(\mexp_1\cdots \mexp_n)$. 
      Then $\EE{e}\rr{X} v$, iff $e\lr v$ and $L[v]\rr{X} v$.
  \end{itemize}
\end{proof}

\begin{theorem}[Correctness]
  For any $\ce \in L'$, 
  \[ \EELL{\ce} \rr{X} v \quad\miff\quad \EEL{\DS{\ce}} \rr{X} \DS{v} \]
\end{theorem}

\begin{proof}
  Proved in lemma \ref{lemma:b-db}.
\end{proof}

\subsubsection{Abstraction}

Correctness is the foundation of the algorithm,
 while abstraction is the aim of our algorithm.
Before the proof of abstraction property,
 we first present \textit{strong abstraction theorem}.

% \begin{definition}[Strong Abstraction]
%   Consider $L'$ is defined by translation rules on the host language $L$.
%   Any constructor used in evaluation rules of $L'$ is value constructor.
% \end{definition}

\begin{theorem}[Strong Abstraction]
  If all the evaluation rules of the host language $L$ are structural,
  the derived rules are also structural.
\end{theorem}

A structural evaluation rule required that the constructors used must be value constructors.

\begin{proof}
  
\end{proof}

Strong abstraction property futher illustrates the importance of structural evaluation rules.
And abstraction allows the language constructs to be used in the rules.
Since the lambda abstraction in the translation rules has been defined as a new translation rules by lambda lifting, 
 the original translation rules are defined by this new language constructs, which is a DSL construct.
Recursively, all generated translation rules satisfy such property.
Therefore:

\begin{theorem}
  Lambda lifting maintains the abstraction.
\end{theorem}



% \subsection{Substitution in Semantic Derivation}


